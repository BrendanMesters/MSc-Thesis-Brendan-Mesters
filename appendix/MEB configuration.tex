\chapter{MEB configurations}
\label{appendix: meb configuration}

The algorithm to calculate a Minimal Enclosing Ball (MEB) with outliers takes three input parameters:
\begin{itemize}
    \item $\delta$, a tightness bound
    \item $\gamma$, a noise percentage
    \item $z$, the number of itterations
\end{itemize}

The algorithm works by iteratively calculating a best guess for the MEB and adding more points to its estimate until it eventually arrives at a reasonable prediction.
The \textit{tightness bound} and \textit{noise percentage} inform the system how lenient it can be when adding new points.

The precise values for these variables, used in the paper, were arrived at by manual tuning of the system.
If the system is deployed in a different environment, then this tuning may not be optimal anymore. In general, $\gamma$ should be tuned to be as low as possible without including any obvious noise points. $z$ should start low and be tuned up, until the resulting estimations do not significantly improve anymore.

For our purposes, we found that taking $\delta$ to be 0.08, $\gamma$ to be 0.15, and $z$ to be 50 results in good estimations roughly 90\% of the time.
We then generate such an estimation 100 times, and average both the ball center and the ball radius across these 100 trials to arrive at a final estimation.
For a mathematical description of the algorithm see \cite[Algorithm 1]{ding2020sublinear}