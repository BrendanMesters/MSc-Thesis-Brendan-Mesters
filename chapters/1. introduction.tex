\chapter{Introduction}
\label{chapter: introduction}


The landscape of Human Computer Interactions (HCI) is shifting as our computers are becoming more powerful and more integrated.
Many new electronic devices are not standalone tools anymore, such as the home computer, these devices are now a part of a broader IoT environment, ranging from smart lights to voice assistants.
Controlling this IoT environment with traditional input methods, such as a mouse, keyboard, or touchscreen, is intrusive and impractical.
Modern alternative input methods are needed for these systems.
Smartphone apps and voice-controlled systems do fill this role,  in the form of smartphone apps and voice commands, but a broader set of options is needed.
This thesis proposes such an interface, which perceives the user's pose through a Millimeter Wave (mmWave) radar, and uses that data to control a computer system.
Similar systems have been built before, using the same sensor technology, but these previous systems relied on deep learning models, which are prone to overfitting, may need extensive retraining in new environments, and require an impractical amount of training data to produce good results.
This thesis sets out to design a Human Pose Estimation (HPE) system, using an mmWave sensor, which relies solely on explainable, debuggable, and tunable algorithms.

\section{Motivation}
\label{section: introduction - motivation}

Computers are getting more and more integrated into our everyday lives, from smart watches to voice-controlled house assistants, from electric cars to IOT houses.
Therefore, it's important to explore robust, intuitive, and power-efficient interfaces to these systems.

\section{Research Challenge}
\label{section: introduction - research challenge}

%     -[ Why mmWave is more difficult then cameras ]-
Using mmWave radars over alternative sensors, such as cameras, introduces unique problems.
Unlike cameras, which provide dense, rich, and low-noise data, mmWave radar pointclouds are sparse (20-30 points per frame at 10 frames per second) and noisier.
Some frames are also exceptionally sparse (~5 points) or contain data only of either the right or left half of the user, which makes those frames less valuable.
This sparseness poses a significant challenge when it comes to \textit{real-time}  systems with a \textit{low-latency} requirement.
In order to have a tight response time, the system needs to generate an accurate response in a single frame or a few frames at most.
Since some frames lack data from a specific section of the user, a response needs to be generated with only a single frame of data, or we risk having a noticeable response delay.

%     -[ Why algorithms are more difficult then DL ]-
The avoidance of DL methods poses another challenge, as they do provide an impressive inherent ability to generalise, even if this is achieved through a \textit{black box}.
The move to an algorithmic solution requires domain-specific prior knowledge to parse the sparse point cloud and infer the actual events that generated that specific input.
This is a difficult challenge when working with high-quality dense data, exacerbated by the noisy nature of mmWave radar point clouds. 
The input data should be fully exploited, using the known physical constraints, in order to observe the underlying events correctly.

%     -[ Why my problem statement is simpler then SOA ]-
However, a full-body skeletal estimation is often unnecessary for a specific interaction task and introduces extra computational load and latency.
To avoid this additional load and any potential dependency on large training sets, the problem space for this thesis is explicitly constrained.
Instead of a skeletal estimation, this thesis provides an \textit{arm angle estimation}, focusing on a fast response time with low overhead, allowing for quick retuning, as opposed to retraining.


% Many real-world applications do not need a full \textit{skeletal estimation}, instead requiring a more contained estimation of the user's actions.
% Thus, in order to avoid dependency on large training sets, the problem space will be constrained for this thesis to an estimation of the \textit{angle of the user's arms}, as opposed to a full \textit{pose estimation}.

%This thesis also focuses on a more constrained problem space, estimating the \textit{angle of the user's arms}, as opposed to doing a full \textit{pose estimation}.



% %        -----------
% %        [ PROJECT ]
% %        -----------
% 
% %     -[[ (TECHNICAL) Clear and concise problem statement ]]-
% For this thesis, we will assess the viability of algorithmic interpretation methods for mmWave radar pointclouds for human pose estimation.
% This will take the form of a HPE interface to a musical application, 
% This thesis will create an algorithmic interpretation algorithm for mmWave radar pointclouds.
% This system will measure the angle of the user's arms in real time
% %     -[[ (BRIEF) How and why music ]]- 
% 
% %     -[[ Itemize research challenges ]]- 
% 
% In this thesis, we will explore the possibility of algorithmic interpretation methods of mmWave point clouds.
% We will design a system which can predict the general position of a users arms, into one of three distinct regions, "low", meaning angled at ...., "middle", which is the region of angles ...., and high, given the region of angles ....
% 
% \textbf{Research Challenge}: 
% \begin{itemize}
    % \item 
    % This thesis aims to design an effective algorithmic interpretation method of mmWave radar pointclouds for the purpose of human pose estimation.
    % \item
    % This interpretation method should be explainable, as well as tunable to changing circumstances.
    % \item
    % This interpretation method should hold up, if not exceed early Deep Learning methods for similar purposes.
% \end{itemize}

\section{Contributions}
\label{section: introduction - contributions}

This thesis has made a few contributions to the field of mmWave HPE, namely:
It provides a system for arm angle prediction; it shows that algorithmic methods have unexplored potential when it comes to HPE; it provides information on successful and unsuccessful methods for algorithmic data enhancement and interpretation of mmWave Pointclouds.

% \input{sections/1. introduction/1.3. scientific gap}
% \section{Background}
\label{section: introduction - background}

\textbf{¿¿¿ Should this be a section or a chapter ???}

This thesis works with a specific sensor, the Millimeter Wave Radar (MMWave Radar), and tries to solve a specific problem, Human Pose Estimation (HPE). 
This section aims to give readers the necessary knowledge on these subjects to be able to read and understand this thesis.
Furthermore, we will discuss some of the current state-of-the-art systems for HPE using an MMWave radar.

% I'll give background on MMWave, HPE, and the current SOA HPE methods used with MMWave.
% I will also discuss some of the properties of MMWave which makes it difficult to handle

\subsection{Millimeter Wave Radar}
\label{sub-section: introduction - background - millimeter wave radar}
Give an explanation of what MMWave radars are, to what extent we can configure them, and what we get out of them (3d pointclouds, at ~10Hz, with between 30-100 points per frame).
Mention the inherent privacy-preserving nature of MMWave due to its not capturing images, and how it fares better in varied light conditions (overexposure/underexposure).
Also, briefly explain how they work internally, using \textit{beam forming antennas} on the receiving and, as well as the large amount of FFT's over the raw signal, to get to pointclouds.

I should mention that the data from MMWave radars is not always temporally stable.
Aka, one frame I might see my left arm, but the next I might see my right.
This is important for section \cref{sub-section: methodology - data filtering - temporal filtering}.

\subsection{Human Pose Estimation}
\label{sub-section: introduction - background - human pose estimation}
Explain what exactly human pose estimation is, and reference some papers discussing it. 
Present the kinect and skeletal estimation, and \textbf{clearly specify} the \textbf{differences} between my \textit{arm location} estimation, and a \textit{full skeletal estimation}.

\subsection{Current State of the Art}
\label{sub-section: introduction - background - current soa}
Briefly mention some \textit{non-mmwave} HPE solutions, those using cameras and those using sensors, as well as their shortcomings (privacy/light situation and clunkiness of wearables, respectively).
Discuss the various SOA MMWave HPE sollutions, their strengths, weaknesses and shortcommings.
Also specify the strengths of Stochastic explainable systems over DL systems (tunable, testable, explainable, significantly lower need for training data), and setup a bridge to the scientific gap.



% \input{sections/1. introduction/1.2. scientific gap}

% \input{sections/1. introduction/1.3. objective}

% \input{sections/1. introduction/1.4. significance of the research}

% \input{sections/1. introduction/1.5. upcomming chapters}
