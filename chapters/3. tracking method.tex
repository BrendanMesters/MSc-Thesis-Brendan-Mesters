\chapter{Tracking Method}
\label{chapter: tracking method}

The main goal of the thesis is to find a method for predicting \textit{where} a user's arms are located, from a pointcloud generated by a millimeter wave radar.
More specifically, it should tell us in what general \textbf{zone} the arm is currently present, this can be either: 
"low", where the user's arm points at the ground;
"middle", where the user's arm points out to the side;
or "high", where the user's arm points up to the sky.
These regions exist both for the left and for the right arm, and thus there are 6 different \textit{zones} which the system tries to predict.
There are, of course, some physical bounds on this model. 
A person can, for example, only have each arm in \textit{one zone}, that means the system will always output \textbf{one} \textit{left zone} (left low, left middle, or left high) and \textbf{one} \textit{right zone}.

To facilitate this zone detection, a standalone pipeline was built to handle both the communication with the FMCW and packet decoding, as well as the filtering, labeling, and interpretation of the data. 
This thesis discusses the latter part of this pipeline, which implements a noise filtering system and our interpretation algorithm.
The final system used by IAmMuse will be discussed in \cref{section: tracking method - data enhancement}
 and \cref{section: tracking method - data interpretation}.
The final pipeline for IAmMuse only arrived after a lot of different approaches had been explored and proved unfruitful.
These approaches are briefly discussed in Appendix \ref{appendix: failed methods}.

% After that, \cref{section: tracking method - different approaches} will discuss some of the earlier attempts, as well as reasoning why those might have failed.

% Why we use an initialization step.
Most of the explored approaches rely on some \textit{training data} of the user.
This training data gives context that is used during the testing.
Interpretation of mmWave data without this context is highly challenging due to the low point density, along with the noise percentage.
This reliance on training data could be remedied in the future by utilizing a greater understanding of these interpretation methods.
However, this thesis still uses training data as it aims to create that \textit{understanding of interpretation methods}.



% The IAmMuse system, as well as most of the failed approaches, uses a workflow where the final \textit{interpretation} system relies on an \textit{initialization} performed by the user.
% During the creation of IAmMuse, it was found that it was not feasible to interpret the MMWave data without any sort of context.
% Such an \textit{initialization} step may not be needed in the future, as the understanding of effective interpretation methods improves, these steps could feasibly be done in real time, during usage.
% However, creating such an understanding of effective methods to generate context from a pointcloud, does benefit greatly from a well-defined initialization step, for that reason, IAmMuse chose to use such an initialization step.

\section{Data filtering}
\label{section: tracking method - data filtering}

One issue present with the mmWave Sensors (in our experience) is the fact that they are susceptible to both environmental noise as well as seemingly random noise. 
This noise often occurred in "streaks", lines of points through your 3d space. 
These "streaks" were not very dense spatially and usually only existed for a single frame, they often formed in the general shape of a line, though never exactly so, see \cref{figure: noise streak}.
Due to these attributes, a combination of a temporal as well as a spatial filter was used to filter out the noise points.
% I could've used a temporal DBSCAN algorithm

\begin{figure}
    \centering
    \includegraphics[width=0.5\linewidth]{figures/internal data/noise streak.png}
    \caption{A noise streak in the data (right-hand side of image)}
    \label{figure: noise streak}
\end{figure}

Many sophisticated systems for pointcloud filtering exist \cite{ester1996dbscan, faisal2017density}, however, most of these systems are built for Lidar, which has a significantly higher number of points. 
Even papers describing methods for "sparse pointclouds" reason about thousands of points at a time, as opposed to the dozens of points encountered in mmWave data.
This makes mmWave pointcloud denoising a truly separate problem, as is evident by the dataset produced specifically for this problem \cite{farella2019sparse} by \citeauthor{farella2019sparse}.
For these reasons, IAmMuse implements its own noise filters, built upon knn.

For this section, it's useful to know that the MMWave produces roughly 10 frames per second, thus each frame is roughly 100 ms.

\subsection{Spatial Filtering}
\label{sub-section: tracking method - data filtering - density filtering}

A common strategy for filtering spatial data is to look at the density and internal cohesion between points.
The main idea on which this is based is the assumption that points with \textit{many close neighbors} are likely to \textit{represent reality well}, a "strength in numbers" kind of philosophy.
IAmMuse uses \textit{spatial filtering} that considers the clustering of points on a \textit{frame-by-frame basis}.
Each point gets assigned a \textit{density score} based on the cumulative weight of its neighbors (points with a distance < 25 cm). 
If their \textit{density score} exceeds a certain (configurable) threshold, then the point is labeled as "signal", otherwise it is labeled as "noise".

The main task of spatial filtering in IAmMuse is to remove outliers, to always let dense clusters pass, and to give an indication of how important any sparse clusters are.
By using the cumulative weight of the neighbors of a point, we also ensure that points that have already been shown to be more likely to be of use by different systems (such as the one described in \cref{sub-section: tracking method - data enhancement - minimal enclosing ball}) will be more likely to be labeled as "signal".
One type of noise that the spatial filtering does not handle well, however, is dense but brief noise spikes.
Since our spatial filtering method only takes into account the point density within a single frame, it's not able to discern between these noise spikes and the actual signal, and that is why the system also uses \textit{temporal filtering}.

% We also filter on the density in a single frame, since high in-frame consistency also often shows high quality of the data.
% However, sometimes noise streaks are dense enough to falsely trigger this system.


\subsection{Temporal Filtering}
\label{sub-section: tracking method - data filtering - temporal filtering}
The main task of the \textit{temporal filter} in IAmMuse is to provide stability across frames and to detect and remove noise spikes.
The temporal filter achieves this in a very similar way to the spatial filter, but instead of looking for neighbors (nearby points) in the same frame, it only considers neighbors in a configurable number of earlier frames.
More specifically, IAmMuse's temporal filter takes into consideration the three frames (0.3-0.4 seconds) that came before, it then gives each point a score equal to the cumulative weight of its neighbors (points within 20 cm) within these prior frames.
This means that points in a dense cloud may all get a very low score if little or no data has been found in that area recently.
In this way, the \textit{temporal filter} can filter out any inconsistent data, such as sudden but dense noise spikes.

Removing inconsistent data is not the only thing that the temporal filter does.
The temporal filter also gives a high score to, and thus selects, points in temporally stable areas more than the density filter would, because the temporal filter takes into account points of multiple frames.
This means that any data that is consistently present in an area across frames will be highly rated by the temporal filter, even though it takes into account a slightly smaller neighborhood (20 cm as opposed to the 25 cm of spatial filters).
That being said, on their own, temporal filters can often miss large amounts of data, since the MMWave sensor does not always produce temporally consistent data, as discussed in \cref{section: background - millimeter wave radar}.
That is why IAmMuse uses a combination of both a \textit{spatial} and a \textit{temporal} filter.


% Temporal data filtering was chosen, as the noise most often encountered was only present in a single frame. 
% By comparing the points in subsequent frames, these noise points can be filtered out by looking at the stability across frames.
% This was implemented by checking the number of neighbors on different frames that any point had within a specified distance (e.g., 10 cm). 
% In this case, even dense clusters on a single frame will be filtered out if this cluster is not present on other frames.
% 
% A danger of this approach is that you might lose correct points, this problem becomes more prevalent if the data is not always consistent over time.
% If the FMCW measures an arm (so not noise) in frames 1, 3, and 5, all that data will be discarded if we only consider a window of 3 frames (centered on the frame to be looked at, so on frame 3, we consider frames 2, 3, 4).
% Due to this side effect, it's important to choose the \textit{window of frames} you consider, as well as the \textit{minimum number of neighbors} and the \textit{distance within which you count as a neighbor}.


\subsection{Combined Filtering and Tuning}
\label{sub-section: tracking method - data filtering - combined filtering and tuning}

IAmMuse uses a combination of the \textit{spatial} and the \textit{temporal} filters, so that it can get the benefits of both while mitigating their weaknesses.
It does this by calculating both the \textit{density score} and the \textit{temporal score} for each point, and combining these, in accordance with an distribution scale. 
This \textit{combined score} is then compared to a threshold value, also composed of the two seperate threshold values combined through a distribution scale.
% where it combines their respective scores, to make use of the strengths of both, while mitigating the weaknesses of both.
Thus this combined filter has six variables which can be tuned: the number of frames considered in the temporal system; for both the \textit{spatial} \textbf{and} the \textit{temporal} filters, the neighborhood size considered, and the threshold value used to distinguish between noise an signal; and lastly the amount we take into account the spatial result or the temporal result, the \textit{distribution scale}.
Combining these two systems, however, does make it more difficult to tune, as there are more moving parts.
Therefore, a specific tuning strategy is used for IAmMuse, to get the system to a reasonable state, afterwards, small tweaks are still made to fine tune the system.
% Therefore, a specific tuning strategy was thought up beforehand to get the system to a reasonable state.
% Afterward, small tweaks were still made, based on what looked good to the visual inspection of the researcher.
% This is the main reason why we made a specific tuning strategy for this system before we started actually tuning it, though some manual tuning was still required in the end.

The tuning strategy which was used for this system was based on two principal ideas.
Firstly, the fact that the spatial filter is best at identifying most actual signal points, even if some noise comes through, this makes it a good starting point when trying to filter the data.
Secondly, the fact that the temporal filter is good at detecting these noise points which get past the spatial filtering, which makes it a good candidate to clean up the output of the spatial filter.
%In the combined filtering these two filters are both tuned to their strength, combined and finetuned to fit together well.
%that the spatial filtering would be best at getting our positive values, while still avoiding some outliers, and that the temporal filter would be able to provide enough of a counterbalance to mark inconsistent points as noise, while keeping the consistent points to be signal.

Our tuning strategy is thus to start by tuning the \textit{spatial filter} on its own in such a way that it captures all points we deem to be part of the signal, this means that it will also capture some points which are deemed to be noise.
The \textit{temporal filter} should then be tuned, once again on its own, in such a way that it must not capture any noise, even if this means missing some points which are part of the signal.
These two filters are then combined (starting with a 50-50 split of effectiveness), and fine-tuning will happen. 
The goal of this fine tuning is to increase the prevalence of the spatial filter if data points are being missed, or to increase the prevalence of the temporal filter if noise is being labeled as data.
This can be done by changing the \textit{distribution scale}, increasing the neighborhood of the filter which is not having enough effect, or decreasing the neighborhood size of the other filter. 
The \textit{threshold value} can also be changed by modifying the two threshold values which make it up, lowering or increasing it to increase or decrease the number of points labeled as data respectively
% v [ old ] v
% Here, the various variables will be modified slowly to resolve problems that are encountered.
% In general, if signal points are missed, we should want either to increase the neighborhood size of the spatial filter, lower its threshold, or make it more prevalent in the filter split.
% In a similar vein, if noise points are not identified as such, we will want to increase the prevalence of the temporal filter in the filter split, lower the temporal threshold, or decrease its neighborhood size.

A big advantage of this system is that it can be quickly tuned to the different scenario, since the output of mmWave sensors can differ quite drastically based on the physical setup of the sensor and its environment.
This means that a system can be moved around with relatively little effort when you compare it to systems based on deep learning.
If you are able to define a good quality-metric for a point cloud recording, then you might even be able to automate this tuning, due to the low number of input parameters to the system (6).

% Combined filters aim to make use of the best of both filters while mitigating the weaknesses of both filters.
% Tuning such a system is always an important part, the way this system was tuned was by first loosely tuning both systems, keeping in mind what we want from them.

% The density filter was tuned in such a way that it would recognize most actual hand recordings (with any significant number of points, e.g. 5-7 min) while letting through as few noise streaks as we can.

% The temporal filter was filtered in such a way that it notices a decent number of hand positions, while prioritizing filtering out noise streaks.

% So density should catch almost all hands, while temporal should ignore almost all noise.

% During the combination of both filters, we want to give enough weight to the temporal filter, to remove most noise streaks, but we want to not make it too great, so that we still recognize hand clusters which are only "good visible" in one frame.

% We tune the individual filters with Distance (for neighbor search) and the threshold neighbor value. And for temporal, we also use the number of frames considered (values between 3-7, where considered, though any number > 5 (~half a second) already risks being more noise than data).
% Lastly, we of course have to decide in what manner we want to combine both.

% The advantage of doing it this way is that you can "retune" if you're tackling a different scenario. e.g. a scenario with slow-moving hands will probably want to consider a higher number of frames, while one with fast-moving hands wants to consider fewer frames. So too you can make density filters more prevalent in your final filter if you don't encounter a lot of random noise, while you can lean more on temporal filters if you do.


\section{Data Enhancement}
\label{section: tracking method - data enhancement}

\begin{figure}[!htb]
    \centering
    \includegraphics[width=0.9\linewidth]{figures/internal data/Thesis zone prediction internal.png}
    \caption{Step by step zone prediction visualization}
    \label{figure: zone prediction internal}
\end{figure}

Data enhancement is the first application-specific step in the prediction pipeline.
In addition to the data filtering discussed above, we also give more weight to points near the hands.
These points are most indicative of the arm angle, as they move the furthest for a change in the arm angle.
To find the points near the hands, a ball is constructed around the pointcloud, where the ball's boundary follows the user's maximal reach.
% This is done by constructing a ball around the pointcloud, where the edge of the ball should follow the reach of the person's arms.
Figure \ref{figure: zone prediction internal} shows a visualization of the steps in the prediction pipeline.
After noise filtering, \textit{arm zones} can be specified as slices of the enclosing ball.
An arm prediction is made based on the number of signal points in each zone.


The data from the mmWave sensor is relatively sparse; thus, it is essential to use this data to the fullest extent.
% get as much information out of this data as we can. 
This process is called \textit{data enhancement}.
Data enhancement broadly covers any method that increases the \textit{quality} of the data sent to the recognition systems.
There are two distinct groups of data enhancement methods.
Physical enhancement, changing the physical setup in order to get better results, and software enhancements, where software is put in place to add extra information to our data.
The first option, a change to the physical setup, can be effective, however, we should avoid this for the final product as it limits the flexibility of the system.
Physical setup changes can help significantly during the design of a system, as high-quality data is needed to design a good classification algorithm, and a good classification algorithm is needed to verify data enhancement methods.
Software-based data enhancement is more challenging to create, but also more flexible, which makes it a preferable option for the final system.
These systems work by adding additional information to the data already present, this is also where the difference between data filtering (\cref{section: tracking method - data filtering}) and data enhancement exists.


\subsection{Physical Enhancement}
\label{sub-section: tracking method - data enhancement - physical enhancement}

\begin{wrapfigure}{!h}{0\textwidth}
    %\centering
    \includegraphics[width=0.28\textwidth]{figures/experimental setup/reflective wearable.png}
    \caption{A reflective wearable}
    \label{fig: reflective glove wearable}
\end{wrapfigure}


% SOA wearables
Many similar HPE systems rely solely on physical wearables \cite{filippeshi2017survey, antonio2010the}. 
This is a logical choice, as it enables the direct tracking of specific parts of the user's body, at the cost of being more invasive and not easily deployable in non-private spaces.
% IAmMuse
Our system aims to give more weight to points belonging to the hands.
Therefore, a \textit{reflective glove} (see \cref{fig: reflective glove wearable}) was chosen, as reflective surfaces produce denser pointclouds with mmWave radars.


\subsection{Initialization, Minimal Enclosing Ball}
\label{sub-section: tracking method - data enhancement - minimal enclosing ball}

The primary goal of our data enhancement system is to make a distinction between points belonging to the hands, points belonging to the arms, and points belonging to neither.
A conceptual ball, which encloses the user in such a way that the surface of the ball follows the maximal range of the user's hands, help in achieving this distinction.
% This distinction is achieved via a conceptual ball, which encloses the user in such a way that the surface of the ball follows the maximal range of the user's hands.
The distance between the surface of the ball and any particular point functions as a proxy for whether a point was emitted by a user's hand, arm, or neither.
To construct this \textit{enclosing ball}, we use the fact that only a minority of the points are noise.
We collect a pointcloud of 3000 points by accumulating recorded points while the user moves their hands up and down.
All collected signal points lie within the proposed enclosing ball, and these points will cover the whole range of movement of the user's arms.
Some noise points will still lie outside of the proposed enclosing ball.
We can now use a known algorithm to construct a minimal enclosing ball from a pointcloud with outliers.




% MEB method
For this, we used a process described by \citeauthor{ding2020sublinear}, in \textit{algorithm 1} of their paper \cite{ding2020sublinear}.
This algorithm gives a stochastic estimation of such a \textit{minimal enclosing ball with outliers}, with bounds on the accuracy.
The stochastic nature of this method results in an inaccurate fit in roughly 10\% of the predictions.
Though somewhat infrequent, this occurs often enough that a mitigation strategy is warranted.
For this reason, the final enclosing ball estimation is constructed as the average of 100 different predictions, to average out the noise and get a stable prediction.
% For this purpose, the enclosing ball is estimated 100 times, and the average of these trials is used as our final estimation.
An in-depth explanation of the configuration of this algorithm can be found in Appendix \ref{appendix: meb configuration}.

% limitation, static ball.
A current limitation of this system is that a user can only \textit{initialize} the system once, and this initialization dictates where the user has to stand, since the system expects the user's sternum to be stationary and at the center of the enclosing ball.
A more dynamic version of the initialization system was explored, where the enclosing ball would be recalculated each frame with the most recent 3000 points.
In this approach, the enclosing ball would drift upwards if the user held their hands high for an extended period of time, invalidating the estimation.
% This approach did not work well as the MEB started to drift at times, depending on the most recent positions the user held.
% As an example, if a user wanted to hold their hands up high a lot, then after enough time, the point cloud used to generate the MEB would not be representative of the user's whole range of movement anymore.
Similar systems, which allow the user to move slightly during execution, can be envisioned but were deemed outside of the scope of this research and are thus left for future research.


\subsection{Programmatic enhancement: point weight}
\label{sub-section: tracking method - data enhancement - point weight}

% What did I tell before:
% - Why the hands are important
% - How programmatic and physical enhancement are similar and different
% - Why the encompassing ball is needed
% - How we use the encompassing ball to calculate how "handy" a point is
\begin{figure}[!htb]
    \centering
    \includegraphics[width=0.7\linewidth]{figures/internal data/hand weight exaggerated.png}
    \caption{Exaggerated visualization of programmatic point weight system}
    \label{figure: exaggerated point weight}
\end{figure}


After training, the points can be given a weight, based on how near to the hands they are, using the \textit{encompassing ball}.
For any given change in arm angle, the points produced by the user's hands will move the furthest out of any of the signal points.
The physical enhancement (\cref{sub-section: tracking method - data enhancement - physical enhancement} generates a higher number of points of the hands to increase the prevalence of that region of the pointcloud.
The programmatic enhancement, on the other hand, adds the dimension of weight to represent this prevalence for each point.

The encompassing ball gives a sufficient amount of context about the scene to be able to estimate which points are generated by the hands, arms, or neither.
This context is not present in any one frame, and must thus be gathered via a brief training step.
Figure \ref{figure: exaggerated point weight} shows an exaggerated visualization of how the weights of the points change, using the encompassing ball.
The points that are near the hands of the ground truth have an increased weight, while most other points have a decreased weight.

This weight change is based on the distance to the surface of the encompassing ball, and the precise weight is determined using the \textit{Gaussian Function}.
The expected value of our Gaussian is equal to the radius of the encompassing ball, and the input to the function is the distance from the center of the encompassing ball to the point in question.
This ensures that the weight of a point is inversely proportional to the distance to the ball's surface.
For the variance ($\sigma^2$), a value of $0.25$ was used; this value ensured a high weight for any signal points belonging to either the hands or the arms, while reducing the weight of points further away by more than half.
Other systems should take into account that the weights generated by this system are always smaller than one.




\section{Data Interpretation}
\label{section: tracking method - data interpretation}

\begin{figure}[h]

    \begin{subfigure}{0.5\textwidth}
        \includegraphics[width=0.9\linewidth]{figures/internal data/IAmMuse internal view.png}
        \caption{Internal data view}
        \label{figure: internal data view, a}
    \end{subfigure}
    \begin{subfigure}{0.5\textwidth}
        \includegraphics[width=0.9\linewidth]{figures/internal data/IAmMuse internal view with zones.png}
        \caption{Internal data view with zones}
        \label{figure: internal data view, b}
    \end{subfigure}
    
    \caption{Internal data view}
    \label{figure: internal data view}
\end{figure}
% \textbf{¡¡¡¡ The explanation is quite difficult to follow currently, think up specific terms and meanings, in order to make the explanation more clear!!!}\\
% Things to get a specific term for
% \begin{itemize}
    % \item MEB center
    % \item MEB radius
    % \item MEB circle (for reasoning)
    % \item specific arm zones, e.g. left low; left mid; left high; right low...
    % \item angle ranges
% \end{itemize}

The method used for data interpretation is tightly linked to the method used for data enhancement, since the data enhancement method adds a specific type of information to our input data.
In the case of IAmMuse, we have the ball, which encompasses the user, as well as the extra dimension of weight, added to the point cloud.
The data interpretation method of IAmMuse will thus be built around this new information.
A visualization of this internal state, with the ground truth in orange, is shown in \cref{figure: internal data view}.


\subsection{Conceptually}
\label{sub-section: tracking method - data interpretation - conceptually}

% What do we want to predict
As was mentioned at the start of \cref{chapter: tracking method}, the goal of IAmMuse is to predict what \textit{zone} a specific hand is in.
These zones are defined as being a \textit{range of angles}, and can be either "low", "middle", or "high". 
Respectively representing situations where the user points at the floor, outwards, or at the sky.
% 2D simplification.
When reasoning about this system, a useful simplification is to think about the system in two dimensions.
This would change our enclosing ball into a circle, enclosing the movement space of the user's arm, where the center of the circle is still at the user's sternum, and the edge still traces the potential positions of the user's hands. 
Each 3D point would then be transformed into its 2D projection, where we remove the direction outwards from the mmWave camera.
The reasoning performed on this simplified model still holds in the actual model, why this is the case will be specified throughout \cref{sub-section: tracking method - data interpretation - practically}.


% pointcloud -> angleset
Since each point can be interpreted as a prediction/reading of the position of the arm, it's also possible to determine a prediction/reading of the angle of the arm, as predicted by that singular point.
This \textit{point angle} can be defined as the angle of the line which goes from the user's sternum (center of the enclosing circle) to that specific point.
This allows us to transform our weighted pointcloud into a set of predicted angles, with a weight equal to the original point weight, which represents the "value" of that particular prediction.
With this list of weighted angles, it's now easy to make some likelihood predictions for specific arm zones by simply looking at the zone in which each angle falls and the weight of that angle.


\subsection{Practically}
\label{sub-section: tracking method - data interpretation - practically}

% Pointcloud -> angleset
The first step that needs to be performed is to transform the weighted point cloud into a set of weighted angles, corresponding to the angle which the arm would have if it went through the specific point.
To calculate this angle, the vector between the center of the enclosing ball and the specific point considered is taken.
Then the arc-tangent of this vector's distance in the X-axis (side to side) and the Z-axis (up-down) is considered to be the angle of the point.
Since the user is expected to hold their arms out mostly straight, the user can be correctly assumed to be roughly two-dimensional.
From this, we can see that the encompassing ball can be viewed as an encompassing circle, since the center of the circle is said to be at the location of the user's sternum, and is thus at the same distance from the mmWave camera as the user.
% \textbf{INCLUDE THIS?}
A better method would have been to consider the angle between the aforementioned vector and the down vector, making a distinction between "left" and "right" based on the sign of the X component of the vector.
This would allow the user to hold their arms somewhat more forward without affecting the accuracy of the system.
In this case, the simplified two-dimensional reasoning still applies, as you can take an X-value in the 2D space as being $\sqrt{x^2 + y^2} \times \text{sign}(x)$.
% \textbf{END OF INCLUDE THIS?}
Doing this for all points in the weighted point cloud yields us a list of angles with an associated weight.

% angleset -> prediction
Each arm zone is internally defined as a range of angles, where a zone should thus be selected if the specific arm is within that specified angle zone.
The way IAmMuse differentiates between the left hand and the right hand is simply by the angle.
These angles exist on a range between $0\degree$ and $360\degree$, where $0\degree$ is down, $90\degree$ is left, $180\degree$ is up, and $270\degree$ is right. 
In this way, systems can be generalized over both arms.
Each arm zone then gets a weight equal to the cumulative weight of all the angles that fall within its angle range, only considering angles from one frame.
The right and left zones with the highest weight are then predicted to be the zone where the user's hands are currently.
Making sure only to select \textit{one} left zone and \textit{one} right zone.

% Why we need stabilization
% This system has a few issues, though. 
% For one, very sparse frames currently have the same effect as very dense frames, while the information present in a sparse frame is, by definition, lower.
% Since the mmWave radar has frames where it simply produces less, and less useful data, it often happens that a small bit of noise will produce an almost random prediction on these frames.
% Even on denser frames, it's possible for large noise spikes to poison a specific prediction.
% Lastly, there is the issue where an arm located at the edge of two zones will be predicted to be one or the other almost at random for any particular frame, due to the inherently random point distribution point samples generated by the mmWave radar.
% While this is not a big deal when it comes to a single frame, during system usage, this means that the system might rapidly change its prediction back and forth, resulting in a non-stable output on a stable input (user arm position), which is a bad thing.
% For these reasons, it's important to consider methods for data stabilization.

\subsection{Data Stabilization}
\label{sub-section: tracking method - data interpretation - data stabilization}

% stabilization intro & minimal points
There are a few issues with the current system, though, all to do with the stability of the resulting prediction, and extra systems have been put in place to mitigate these issues
Firstly, one issue occurred when the mmWave radar produced very sparse frames. 
Sparse frames contain disproportionately few points, and thus also contain disproportionately few signal points. 
This means that the predictions of IAmMuse are more susceptible to the random variations in the point distributions, as well as being more susceptible to noise, since the prediction system simply looks at which zone has the highest weight.
The way in which IAmMuse mitigates this issue is by setting a \textit{minimal weight} which a zone must have to be considered as a "valid prediction", if no valid prediction was made for a side, the previous prediction is kept.
This mitigation strategy does have the side effect of ignoring some frames if they are too sparse, but this isn't a huge problem, as the amount of information present in that frame was already very low, by way of how sparse it was.
A solution that adds the data from these "missed frames" to the following frames could be made.
This would increase the amount of information the system uses, but could also induce a lower responsiveness by taking into account stale information.
For these reasons, such a system was not put in place in IAmMuse.


% Multiple predictions needed
Another similar issue occurs when a large spike of noise is perceived for a single frame, this influx of faulty data can make the system select an incorrect zone.
These large noise spikes do not happen regularly, but they happen often enough that a mitigation system was useful.
In these cases, most frames in a sequence will give a similar result, but one frame will give a "faulty prediction".
Since we know that, in a correct set of frames, the same zone would likely be predicted multiple times in a row, as the user's arm stays in that zone, we can filter out the faulty case by only choosing a new zone if that zone has been predicted twice in a row.
In this system, it's important not to count "missing predictions" (as specified in the previous paragraph) as a part of these consecutive predictions.
This system does add some delay to the selection of a new zone, specifically 100ms if you receive correct predictions every frame. 
For the IAmMuse system, this delay is a worthwhile tradeoff for the added stability against both noise and random fluctuations in the mmWave radars' output, but this may not be the case for every system.

% Zone expansion
Lastly, IAmMuse had to solve an issue in the scenario where a user is holding their arm near the border of two zones.
In this case, IAmMuse may predict either zone at random, purely due to the random nature in which the points in a mmWave frame are distributed.
While a singular frame prediction would not be incorrect per se, the fact that the system produces an unstable output while the user provides a stable input (a stationary arm) is incorrect.
The way that IAmMuse solves this issue is by expanding the region assigned to the currently selected zone.
This eliminates the issue, as the randomly distributed points that are on the edge of the two zones will be predominantly in the active zone if the zone's angle range is increased.
In IAmMuse, the active zone's angle range is increased by $10\degree$, or $5\degree$ on each side, and neighboring zones are shrunk in accordance.
This increase should be as small as possible while still providing the stability-increasing effect.
During system design, an increase of $10\degree$ seemed to be the smallest amount that fulfilled this wish.





% The data interpretation method is often tightly linked to the data enhancement method used, as we will use the extra context provided by the data enhancement in the interpretation.
% In the case of IAmMuse's final system, the extra context we've been given is the MEB, providing us with a rough center point of the user, as well as an estimation of the outer reach of their arms.
% % simplify to 2d explained
% During the design and explanation of the interpretation algorithm, we will often simplify the user's points to be on a plane, rather than in 3d space. 
% The MEB would then likewise be considered to be a circle, with the center still at the user's sternum, and the edge tracing the path the user's hands can trace (assuming outstretched arms).
% This estimation makes the reasoning a lot more intuitive. At the end of the section, we will also briefly touch on why this reasoning also extends to our actual system, which does operate in 3d space, rather than 2d space.
% % continue
% The main idea which is used in the interpretation of the data is that the angle of the arm is very closely modeled by looking at the angle between the sternum and a user's hand.
% The sternum position is already known to be roughly at the center of the MEB, and the user's hands are known to be roughly at the edge of the MEB circle.
% This transforms our problem of "find the angle of the arms" to "find the position of the hands on our circle", which turns out to be a lot simpler.
% 
% To find the angle of the arms, a few things need to be done: points that belong to the hand should be specified, these points should be placed on the MEB circle, the angles between these points and the MEB center should be calculated, and this set of angles should be transformed into a hand/arm angle prediction.
% The first step is done in a somewhat different way. 
% We know that the points at the circle edge tell us a lot of things about the position of the hands, and the further away a point is from that circle edge, the less we trust the information. 
% Since your arm position tells you something of where the hand is, but your torso won't tell you anything.
% For this reason, each point is given a \textit{weight}, dependent on its distance to the circle edge.
% In practice, this is achieved by taking the distance between a point and the MEB center, and applying the Gaussian function to this distance, where $\mu$ is the radius of the MEB, $\sigma^2$ is taken to be $0.25$, and x is the previously mentioned distance.
% The output of this function is used as a weight of the point, which is representative of "how hand-like" the point is.
% Calculating the angle for each point is a trivial trigonometric function, and will leave us with a set of angles with weights.
% 
% The last part of the interpretation is turning this list of angels and weights into two final arm positions, one for the right arm and one for the left arm.
% In this thesis, we do not care about the exact arm position, rather, we care about the "region" where the arm currently is.
% This region is specified as being "low", "middle", or "high", which are simply ranges of angles in which the arm can be.
% The fact that the system is expected to generate an output region can be used in the interpretation.
% To find out the likelihood of an arm being present in a specific region, you can look at the "weight" of said region, where a region's weight is the sum of the weights of all points (angles) that exist in that region.
% In this manner, a frame-specific prediction for the arm location can be generated by selecting the zone with the highest weight on each side (left, right).
% One problem with this approach is that the output is very unstable.
% 
% A few methods are used to stabilize the output of the arm location predictions.\\
% Firstly, the arm position (low, mid, high) that is currently selected will increase the angle range that it considers "its region" by $5\degree$ on each side. 
% e.g. the middle zone usually considers points "in its zone" when their angle is between $65\degree$ and $105\degree$, but if the middle zone is already selected, it will now consider angles between $60\degree$ and $110\degree$, this region increase will shrink the other regions.
% This system avoids the situations where a user's arm is "on the edge" of two zones and rapidly changes which zone is predicted while the user keeps their arm still.\\
% Secondly, there is a certain threshold weight that a zone needs to achieve to be considered as a "valid zone contender", this makes sure that very sparse frames can not trigger a faulty zone change based on only a few points.\\
% Lastly, a specific zone is only selected as the actual zone if it was predicted to be the correct zone twice in a row.
% Note that some frames may not have any predictions at all if none of the zones crossed the weight threshold, in this case, no prediction was made, thus it does not impede a zone from being chosen twice.
% e.g. \textit{frame n}, zone left low is selected. \textit{frame n+1} no zone is selected. \textit{frame n + 2} zone left low is selected.
% In this situation, \textit{left low} has been selected twice in a row, thus, the zone gets set to left low.
% 
% Earlier in the section, it was stated that we could reason about the minimal enclosing ball as if it were a circle, the reasons for this assumption will be given here.
% First off, during the usage of the system, the user is expected to stand facing the FMCW, with their arms out to the side, lifting or lowering them.
% In this situation, the point cloud produced by the user falls roughly in a plane where we simply ignore the y direction, thus, for the points belonging to the user, the simplification to a circle is valid.
% Furthermore, the number of noise points that still exist in this part of the pipeline is small, and given the spatial distribution observed from these points, the likelihood of one of these points being present near the edge of the ball but in front of or behind the user is very slim.
% Therefore, only outlier points will ever appear near the edge of the ball while not being well modeled by the circle.
% Points that are at a large distance from the edge of the ball will be disregarded automatically by being assigned a low weight, as the distance between the centroid and the specific point can be (and is) calculated in 3d.
% Due to the above given reasons, there is no feasible situation in which the assumption (for reasoning purposes) of a circle does not apply accurately to our actual ball.
% Lastly, it's important to mention how we get the "angle" between two points now, given that we are not actually using 2d coordinates in the system.
% The angle of the arm is considered to be the angle between the down vector and the vector that goes from the MEB circle to the point in question.
% This has the added benefit that users are allowed to hold their arms at a somewhat forward angle if this is more comfortable for them, and the calculation still holds true in the same manner.



% % This section describes HOW we do the data interpretation, given the different systems described above, specifically the minimal enclosing ball.
% % This includes the general high-level overview of the system, including zones on the MEB, filtering, and processing.
% % It will also discuss different smaller systems, such as the active zone size increasing code, the stabilization code, which requires a zone to be selected more often, and will discuss the concept of "weight" in the various system contexts.
% % 
% % \textbf{We use a ball, not a circle, explain}