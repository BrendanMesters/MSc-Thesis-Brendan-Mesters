\chapter{Tracking Method}
\label{chapter: tracking method}


For this thesis, a standalone pipeline was built to handle the whole process from communication with the FMCW and package decoding, through to the filtering, labeling, and interpretation of the data. 
In this report, we will focus on the latter part of this pipeline, which implements this paper's interpretation algorithm.
The final system used by IAmMuse will be discussed in \cref{section: methodology - data enhancement} and \cref{section: methodology - data interpretation}.
After that, \cref{section: methodology - different approaches} will discuss some of the earlier attempts, as well as reasoning why those might have failed.

The IAmMuse system, as well as most later approaches, uses a workflow where the final \textit{interpretation} system relies on an \textit{initialization} performed by the user.
During the creation of IAmMuse, it was found that it was not feasible to interpret the MMWave data without any sort of context.
Such an \textit{initialization} step may not be needed in the future, as the understanding of effective interpretation methods improves, these steps could feasibly be done in real time, during usage.
However, creating such an understanding of effective methods of generating context of a pointcloud does benefit greatly from a well-defined initialization step, for that reason, IAmMuse chose to use such an initialization step.
