\chapter{Evaluation}
\label{chapter: evaluation}

% brief introduction to \textbf{evaluation}, give a small overview "testing against different models, to see the strengths and weaknesses of IAmMuse"

To assess the effectiveness of the methods proposed in this thesis, a comparison should be made with a similar SOA model.
For this thesis, the \textit{MARS} system \cite{an2021mars} was chosen, this choice is substantiated in \cref{section: setup baseline dataset - baseline analysis}.
A few terms should be defined before discussing the specific evaluation metrics used.
% Before discussing the evaluation methods, a few terms should be defined. 
The different positions a single arm can hold: "low", "middle", or "high" are referred to as  "\textit{arm positions}".
The combined position of both arms, however, is called a "\textit{position state}".
With each arm being able to take one of three \textit{arm positions}, that thus gives us \textbf{nine} total \textit{position states}.
%The combinations of two of these arm positions, with nine total options, three for the right arm and three for the left, are called "\textbf{position states}".

The following evaluations will compare IAmMuse and the MARS models on a few different metrics.
One metric, which is used as a first impression of a model's performance, is the accuracy in predicting the \textit{position state}, and the accuracy in predicting the \textit{arm position} of each arm individually, where a random model should see an accuracy of $1 / 9$ and $1 / 3$ respectively.
The likelihood and accuracy of the specific \textit{position states} are also considered.
An important distinction here is that the \textit{likelihood} tells us the chance of a certain system predicting a position state, without considering if that prediction was correct or incorrect.
For these \textit{likelihood} graphs, the ground truth (the actual likelihood distribution) is also provided.
The \textit{accuracy} of a \textit{position state}, however, tells you the accuracy of a specific model, given that the ground truth is that particular \textit{position state}.
The \textit{likelihood} graphs can give insight into bias, while the \textit{accuracy} graph tells us about specific strengths or weaknesses of models.
Lastly, other data visualizations may be used. 
For MARS, the skeletal estimates will at times be shown, which should be self-explanatory.
A "time prediction plot" will, however, also be used at times, this plot shows the ground truth and the prediction for both \textit{arm positions} of a specific system, over time.
This is used to visualize in what ways the systems perform good and bad, in the temporality.

% For this evaluation, a few different metrics will be considered.
% Firstly, the total accuracy will be considered, defined as $\text{total accuracy} = \frac{\text{correct guesses}}{\text{total guesses}}$.
% Secondly, the distribution of predicted position states for a model, versus the distribution of prediction states that the ground truth suggests.
% This should give insights into potential systemic biases in a specific prediction system.
% Thirdly, the accuracy for each of the nine position states for a specific model, to analyze if any particular position is hard to predict.
% In addition to this, some other diagrams or figures might be used; these will be more clearly specified when they are.


\section{Baseline Analysis}
\label{section: evaluation - baseline analysis}

% Apples to Apples
Introduce the apples-to-apples setup,
Explain why training on the calibration of a single recording is the \textit{most fair assessment}.
Briefly explain the training parameters.

% Results
Show the results of the two systems.
Show the accuracy plots
Show a few particularly bad frames of the visual replay.

% Discussion
Discuss why this gives such a bad result.
Give a small reason why this may not be a fully fair comparison (MARS is trained once, IAmMuse gets initialized every time)
Introduce the next section.


%Show the results of mars trained on the calib data of ONLY that specific recording


\section{Larger Models}
\label{section: evaluation - larger models}

% Larger models intro
Refer back to the issues of the last section
Propose two new models
MARS paper and all calibration.
Explain why these two are interesting to look at (original model, a version with the same number of input frames as the original MARS).

% Results MARS-paper
Show the results MARS paper, 
MARS-paper is just trained on different data (configuration, physical setup).
Mention that the config was not specified in the paper.
Bring up the issue of "overfitting" in DL models.

% Results MARS-All-calib
Show the results of all-calib
Discuss its unique "optimization strategy" of T-posing.
Touch on "unrepresentative data"
Touch on the "local minima" issue of DL models.
show images to support my statement (and maybe a diagram showing that T-posing indeed gives minimal loss)

% Discussion
Mention that, even with sufficient training data, DL models can perform poorly
Discuss the need for careful data-set construction
Discuss the need for data sets in the first place.


% Describe the new MARS-trained models we'll be evaluating, (\texttt{mars\_paper} and mars trained on \textbf{all} calib data)

% Show their results, and explain why they perform badly (overfitting, non-representative data).



\section{IAmMuse vs Best MARS}
\label{section: evaluation - iammuse vs best mars}

% Describe how we'll train the best MARS model we could make.
The final set of MARS-trained models, which will be considered, is those that are also trained on the \textit{usage data}.
Two models will be considered, the first one is trained on the usage data of all \textit{standard recordings} and is evaluated on the usage data of all \textit{free play} recordings, this model is called \textit{MARS standard to free}.
In this way, the generalization of the standardized movement set to more realistic usage can be measured, since the free play recordings emulate that very well.
The second model, which will be considered, is trained on the \textit{first and second} standard recordings of each user, and evaluated on the \textit{third} standard recording of each user, this model is called \textit{MARS partial standard}.
This gives a more general view of how well the system generalizes on very similar data.
The reason that the train-test split was made with the same users in both sets is to avoid a situation where an exceptional user (e.g., a very small one) is in the test set, but not in the train set. 
A sufficiently well-created training set should account for it, thus, the "best case scenario" for MARS should also have that present.


% Show the accuracy results, discuss how it is better than the previous MARS, but still worse than IAmMuse
This training method finally produces some reasonable results, which are immediately obvious when looking at the accuracy of the arm predictions, in \cref{figure: best mars arm accuracy}.
Note that the previous MARS models, which were considered, had an accuracy of $<0.4$ for an individual arm, as opposed to the 0.65+ accuracy of these new MARS models.
The combined accuracy has also increased significantly, up to 0.45+ from less than 0.2
Even with all these predictions, however, IAmMuse still performs better in this metric.


\begin{figure}[h]
    \centering
    \includegraphics[width=0.8\linewidth]{figures/results/best mars arm accuracy.png}
    \caption{Accuracy of predicting one or both arms correctly for the IAmMuse system, the MARS standard to free, and the MARS partial standard systems.}
    \label{figure: best mars arm accuracy}
\end{figure}

Looking at the accuracy per specific position state, see \cref{figure: best mars state accuracy}, reveals that the MARS models perform quite well on the positions where both arms are at a similar position, and perform significantly worse in the situations where one arm is "high" and the other is "low".
Meanwhile, the performance of IAmMuse does differ slightly between the different positions, but it stays a lot more tightly, in a range between 0.65 and 0.8.
The reason for this becomes more obvious when considering \cref{figure: best mars state likelihood}.
Both MARS models predict the position state of both hands middle significantly more often, while rarely predicting the position state of one hand low and one hand high, similar to how MARS calib trained did, in \cref{section: evaluation - larger models}.
This is most likely caused by the fact that the mmWave radar does not always produce data of the user's left and right half at the same time, as discussed in \cref{section: background - millimeter wave radar}, and by the fact that some frames are simply very sparse.
On these "low quality frames", the MARS model will default to a \textit{minimal MAE position}, aka holding both hands out straight.

\begin{figure}[!htb]
    \centering
    \includegraphics[width=0.8\linewidth]{figures/results/best mars state accuracy.png}
    \caption{Accuracy per state position for the IAmMuse system, the MARS standard to free, and the MARS partial standard systems.}
    \label{figure: best mars state accuracy}
\end{figure}
\vfill
\begin{figure}[!htb]
    \centering
    \includegraphics[width=0.8\linewidth]{figures/results/best mars state likelihood.png}
    \caption{Likelihood of position states for IAmMuse, various MARS models, and
the ground truth}
    \label{figure: best mars state likelihood}
\end{figure}

% Show the time-prediction diagram (showcasing the "stuttering" of MARS)
This tendency to default to holding both hands in the middle position can be more clearly seen in \cref{figure: time prediction plots}, where the MARS prediction often jumps back and forth between different predictions in rapid succession.
The researchers tried to see if this issue could be solved by applying an averaging smoothing filter with a window size of 7 (the line specified as 'LP'), but even with this extra assistance, the error of the MARS systems was significantly higher than the error of the IAmMuse system.
This issue is most likely in part due to the fact that MARS is not a temporal system, it simply predicts each frame on its own.


\begin{figure}[!htb]
    \centering
    \begin{subfigure}{0.8\textwidth}
        \centering
        \includegraphics[width=1.\linewidth]{figures/results/time prediction plot standard to free.png}
        \caption{MARS standard to free}
        \label{figure: time prediction standard to free}
    \end{subfigure}
    \begin{subfigure}{0.8\textwidth}
        \centering
        \includegraphics[width=1.\linewidth]{figures/results/time prediction plot partial standard.png}
        \caption{MARS partial standard}
        \label{figure: time prediction partial standard}
    \end{subfigure}
    \caption{time-prediction plots for IAmMuse (blue), MARS standard to free (green), MARS partial standard (purple), with the ground truth (red) and a smoothing for the MARS (LP) }
    \label{figure: time prediction plots}
\end{figure}


% Mention that, even when given all the tools we could, MARS could not perform better than IAmMuse
This shows that, even if MARS is given all the potential benefits available, IAmMuse still outperforms it by a significant margin.
Although the MARS models produce a reasonable output most of the time, they still suffer from having a statistical bias to certain positions, and their output can not be stabilized by a smoothing algorithm because the system does not produce a "reasonable output" often enough.
Therefore, it seems clear that, in this problem space, MARS is outperformed by IAmMuse.



% This is seemingly not present (at least, there is no performance drop if you run a system
% trained on tall people on a short person).
% \section{Representative data issue}
\label{section: evaluation - representative data issue}

\textbf{IDEA}: \textit{See if a model trained on the data of the taller users, evaluated on the data of the shortest user performs significantly worse. Only do this section if that is the case}

% intro
Point back to representative data issues in section \cref{section: evaluation - larger models}
Discuss the often "hidden issues" with datasets, 
Discuss why this is less of an issue with "designed systems" (A researcher made explicit assumptions, instead of implicit, which are easier to scrutinize.

% Experiment proposal
Introduce the idea of a "length separation" in your data (test set vs actual users)
Clearly specify training and testing data.
Theorize about results

% Results
Show the results, 
Compare the results with those achieved in \cref{section: evaluation - iammuse vs best mars}
Show that the result is significantly worse
Also, look at the results of the small user vs the longer users for the IAmMuse system, and show that this pattern (of worse performance) is not present there.

% Discussion
Mention how it's indeed easy to "miss" some crucial part in DL systems.
Also mention how algorithmic systems can "relatively easily" patch their system.










\section{Key Takeaways}
\label{section: evaluation - key takeaways}

%     -[[ Base Comparison - Training Data ]]-
In this chapter, a few different trained MARS models have been compared with IAmMuse, none of them could outperform IAmMuse.
The models that produced a reasonable output also needed a significant amount of training data to achieve this, as can be seen in \cref{figure: accuracy vs training data}, which plots the accuracy of a model against the amount of training/initialization data.
This clearly shows that the MARS models do get better with more training data, however, it also shows that a well-designed algorithm (IAmMuse) can outperform the DL systems, even with little to no training/calibration.


\begin{figure}[!htb]
    \centering
    \includegraphics[width=0.8\linewidth]{figures/results/_accuracy vs training data.png}
    \caption{The accuracy for each model considered (y-axis) plotted against the amount of data it used (x-axis)}
    \label{figure: accuracy vs training data}
\end{figure}

%     -[[ Specific pitfalls of MARS models ]]-
% BASELINE
\textit{MARS baseline}, which was trained on the calibration data of a single recording, performed very badly, as was to be expected, due to the small training set.
% PAPER
\textit{MARS paper} did not perform significantly better, even though it had a larger training data set. 
This was most likely in part due to the fact that the specific data recording setup used to create that model was different from the one used in IAmMuse, something Deep Learning models are often sensitive to.
% FULL CALIB
\textit{MARS full calib} also yielded a bad result, the model seemed to get stuck in a local minima during training. 
It held a static position, which minimized the error, as opposed to trying to dynamically interpret the input data.
This was likely in part due to the specific properties of the calibration data.
The models where MARS was trained on a significant amount of \textit{usage data} (\textit{MARS standard to free} and \textit{MARS partial standard}) performed the best. 
These models were, however, also not perfect, as their output was quite stuttery.
% These models still suffered from the fact that they were very stuttery.
This was almost certainly caused by the stability issue with data from the mmWave Radar, which is discussed in \cref{section: background - millimeter wave radar}.

%     -[[ Conclusion ]]-
The IAmMuse system uses significantly less data and produces significantly better results than similar DL systems.
Granted, IAmMuse tries to solve a more specific problem than MARS, but in many cases, a solid application-specific solution is better than a flaky general-purpose solution.
Many of the current shortcomings of IAmMuse, such as the need for calibration and the limited arm angles, are also not inherent and are likely to be solvable with follow-up research, this is further discussed in \cref{section: conclusion - future work}.
Al in al IAmMuse succeeded in showing the potential of \textbf{algorithmic} human pose estimation systems for millimeter wave radar data.
It was able to circumnavigate the low point density inherent to these chips by using domain knowledge to enhance the data, as well as using other tunable systems to solve issues that were encountered.
The various parts of the IAmMuse pipeline can also be inspected and tuned individually, making it feasible to move this system between different usage environments.

Thus, this thesis has succeeded in all the research challenges that were set out.
\begin{itemize}
    \item 
        An effective algorithmic interpretation method for mmWave radar pointclouds has been created. 
        This method interprets these pointclouds for use in Human Pose Estimation application.
    \item  
        The method created is explainable, as is shown by the explanation provided in \cref{chapter: tracking method}. 
        Furthermore, the system can also be tuned to work in changing circumstances.
    \item 
        The interpretation method provided exceeds the performance of a similar deep learning system, in the form of MARS.
\end{itemize}
With that, the authors believe this research to be a success.


% Mention performance of various Mars models
% With equal training, MARS performs super poorly
% Mars fell into a local maxima with config train
% Well-trained is better over the board
% Well-trained still suffers from similar issues as config train

% Start making a conclusion
% MARS had issues with {stability, certain positions}
% This is most likely due to
% IAmMuse didn't
% Only issue IAmMuse had was {zone specification}.

% Thus, IAmMuse is better.
