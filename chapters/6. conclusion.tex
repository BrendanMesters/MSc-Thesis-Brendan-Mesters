\chapter{Conclusion}
\label{chapter:conclusion}

\textbf{After thought/lookback on paper}

% motivation
Currently, the field of HPE is dominated by Deep Learning systems
Algorithmic approaches are new and far between, even though they can be very successful \cite{li2025acoustic}.
This paper explored the possibility of algorithmic HPE for mmWave radars.

% Question
To do this, a pipeline was set up for a musical application
The system specifically tries to predict an arm zone
Results were compared with a similarly sized DL model, MARS \cite{an2021mars}

% Results and limitations
Similar train/initialize data, MARS performs terribly
With good training data, it performs reasonably (though not good)
Limitation: MARS predicts skeletal estimate IAmMuse only arm zone
Scenario-specific solutions are not an inherent downside.
Limitation: In free-play users may have adjusted their exact arm position if they didn't get the note they wanted, making it so they could react to the feedback IAmMuse gave them, without them being able to react to the feedback of mars. (given the instability of mars I don't think this is a huge issue)


% Briefly conclude the whole research
% Give a \textbf{one liner} conclusion
% Refer to our results section, the discussion, and the recommendations


% % \section{Findings}
% \label{section: conclusion - findings}
% Quickly summarize the main findings which were discussed in \cref{section: discussion - implications}.

\section{Results}
\label{section: conclusion - results}

% recap
Briefly mention the core concepts of IAmMuse's system
Mention the research question again (DL vs algorithm)
Briefly mention the difference between "arm angle estimations" and "skeletal estimations", and the way we get arm angle estimations from both

% Results
Mention the failures of the simple systems, and thus their exclusion.
Briefly mention some error numbers, and show that, even with best attempts, IAmMuse outperforms MARS.
"Even with the most help possible IAmMuse outperformed"
If applicable mention \cref{section: evaluation - representative data issue}, and its findings

% Final verdict
IAmMuse outperforms a similar DL model in the problem space setup here.
Repeat our conclusion: (something along the lines of "Its good, but this is just the start, it should not be ignored")


% Mention how it outperforms a similar DL system (MARS), even if we try to help the DL system as much as possible.

% Mention the more limited scope of IAmMuse (arm position, as opposed to skeletal estimate).

% Quickly summarize findings: algorithmic interpretation works; its tunable and more stable; 




% \section{Limitations}
\label{section: conclusion - limitations}

% Simpler Problem space
IAmMuse estimates one of three "arm zones" per arm.
MARS (and most systems) create a skeletal estimation.
This is an inherently different problem.
Note that this simpler problem space is often not a concern for application-specific solutions.

% Translation issues
We did not use a DL model, which predicted "zones"
This added complexity may have caused MARS to perform worse.




\section{Recommendations/Future work}
\label{section: conclusion - future work}
Depending on the results, I either want to encourage more research into stochastic MMWave interpretation, both in breath and scope. aka, people should try and make stochastic systems for more complex problems/different problems. And people should look further at optimizations in the current system, in firmware used, system tuning, and potential broadening.
An interesting problem would be to see if you can increase the amount of zones, and reduce them in size, in such a way that you have ~20 zones per side, and to then try and modify the system to still predict zones correctly.
This could be achieved with a kind of "decaying charge" (each step the zone loses a percentage of its value) and a kernel/distance based "charge addition" (each point gives value to multiple zones, depending on how "near to it" they are).


\subsection{Dynamic Initialization Recalculation}
The initialization step is currently fully static, this could be adapted to become a more dynamic system.

\subsection{More Continuous Result Angles}
One shortcoming of the current system is that it only considers some broad angle-zones.
This isn't a weakness in and of itself, but it limits the usefulness of the system.
One way to potentially mitigate this issue is to change the data interpretation method described in \cref{section: tracking method - data interpretation} to allow for a broader range of output values.
One manner in which that could be done is to consider 360 "zones" of $1\degree$ each. 
Due to the relatively low point density, this system would need to solve the issue that it's very unlikely for multiple points to end up in the same zone (which would make the system prone to random data distribution artifacts), and the fact that the current stabilization methods would not work anymore.
One way to solve the first issue would to to have each point add its "weight" to many zones "near" it, for example, by using a Gaussian kernel to distribute its weight across nearby zones.
This would produce a likelihood-density distribution along the angle axis.
The peaks would be representative of where the arms might be, some clever new stabilization methods would be needed for this system, to ensure that "one sided frames" (frames whose data is predominantly on one side) don't mess things up, as well as to stabilize the output in general.

One potential idea is to add the new weights to a running sum (for each zone) and normalize the output on a specific "Total weight", e.g., at the end of the normalization the sum total of all zone weights should always be 30.

However, the further design and testing of these systems will be left to any future researchers.

