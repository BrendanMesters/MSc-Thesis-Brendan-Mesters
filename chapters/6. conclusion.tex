\chapter{Conclusion}
\label{chapter:conclusion}

%\textbf{After thought/lookback on paper}

%     -[[ Motivation ]]-
% Why did we do this research
In the field of mmWave radar-based HPE, all published solutions use deep learning for data interpretation, which comes with an inherent cost.
Deep learning has two inherent costs: the need for large, representative training datasets and the known high energy costs of deep learning models.
Algorithmic solutions do not have these inherent costs and have been shown to work in other mmWave radar-based systems (\textbf{REFERENCES}).

%     -[[ Research Question ]]-
% What did we want to accomplish
This thesis explores the 

\brendanLines

% motivation
Currently, the field of HPE is dominated by Deep Learning systems
Algorithmic approaches are new and far between, even though they can be very successful \cite{li2025acoustic}.
This paper explored the possibility of algorithmic HPE for mmWave radars.

% Question
To do this, a pipeline was set up for a musical application
The system specifically tries to predict an arm zone
Results were compared with a similarly sized DL model, MARS \cite{an2021mars}

% Why/What we found

% Results and limitations
Similar train/initialize data, MARS performs terribly
With good training data, it performs reasonably (though not good)
Limitation: MARS predicts skeletal estimate IAmMuse only arm zone
Scenario-specific solutions are not an inherent downside.
Limitation: In free-play users may have adjusted their exact arm position if they didn't get the note they wanted, making it so they could react to the feedback IAmMuse gave them, without them being able to react to the feedback of mars. (given the instability of mars I don't think this is a huge issue)


% Briefly conclude the whole research
% Give a \textbf{one liner} conclusion
% Refer to our results section, the discussion, and the recommendations


% % \section{Findings}
% \label{section: conclusion - findings}
% Quickly summarize the main findings which were discussed in \cref{section: discussion - implications}.

\section{Results}
\label{section: conclusion - results}

% recap
Briefly mention the core concepts of IAmMuse's system
Mention the research question again (DL vs algorithm)
Briefly mention the difference between "arm angle estimations" and "skeletal estimations", and the way we get arm angle estimations from both

% Results
Mention the failures of the simple systems, and thus their exclusion.
Briefly mention some error numbers, and show that, even with best attempts, IAmMuse outperforms MARS.
"Even with the most help possible IAmMuse outperformed"
If applicable mention \cref{section: evaluation - representative data issue}, and its findings

% Final verdict
IAmMuse outperforms a similar DL model in the problem space setup here.
Repeat our conclusion: (something along the lines of "Its good, but this is just the start, it should not be ignored")


% Mention how it outperforms a similar DL system (MARS), even if we try to help the DL system as much as possible.

% Mention the more limited scope of IAmMuse (arm position, as opposed to skeletal estimate).

% Quickly summarize findings: algorithmic interpretation works; its tunable and more stable; 




% \section{Limitations}
\label{section: conclusion - limitations}

% Simpler Problem space
IAmMuse estimates one of three "arm zones" per arm.
MARS (and most systems) create a skeletal estimation.
This is an inherently different problem.
Note that this simpler problem space is often not a concern for application-specific solutions.

% Translation issues
We did not use a DL model, which predicted "zones"
This added complexity may have caused MARS to perform worse.




\input{sections/6. conclusion/6.3. future work}

