\section{Motivation}
\label{section: introduction - motivation}


%     -[ Current state of IoT interfaces ]-
The landscape of Human Computer Interactions (HCI) is shifting as our computers are becoming more powerful and more integrated.
Standalone systems, such as the home computer, are becoming increasingly rare.
Devices are now often part of a broader IoT environment, ranging from smart lights to voice assistants.
Controlling this IoT environment with traditional input methods, such as a mouse, keyboard, or touchscreen, is intrusive and impractical.
These IoT environments require new and alternative input methods.
Smartphone apps and voice-controlled systems do fill this role, but a broader set of input options would be beneficial.

%     -[ Why mmWave is a great option ]-
A sufficiently unintrusive interface would not require any physical device to interact with it.
Simple physical movements should be sufficient to control the underlying system.
These types of Human Pose Estimation (HPE) systems have already been explored, but primarily used cameras for the recognition of a person, which are privacy-intrusive and sensitive to lighting conditions.
Millimeter Wave (mmWave) radars, on the other hand, offer a privacy-preserving sensing method that is well-suited for this exact purpose.

%     -[ SOA uses DL, why might algorithms be better ]-
Existing mmWave radar-based HPE systems, which can be used as a computer interface, are predominantly built on a Deep Learning (DL) foundation.
Common limitations of DL systems are their computational intensity, functional opaqueness, and difficulty of tuning.
Algorithmic HPE interpretation systems would be able to run on more affordable hardware while only requiring retuning as opposed to retraining.

%     -[ Why we use musical interfaces ]-
HPE solutions fill a niche of low-latency and high-precision interfaces, for which existing technologies like voice assistants are not suited.
A suitable benchmark system is required to validate such a \textit{real-time} HPE solution.
Musical interfaces are such a system, due to their low latency requirements.
Their immediate auditory feedback further supports this suitability, since jitter and high latency make instruments unplayable.
For these reasons, musical interfaces are an ideal use case through which such an HPE system can be designed.



% %        -------
% %        [ GAP ]
% %        -------
% 
% %     -[[ Why HPE ]]-
% Human Pose Estimation (HPE) is a promising new interface to computer systems.
% The ability to capture a person's stance and use it as input opens the way to a myriad of control systems.
% Most modern HPE solutions use cameras \cite{munaro2014-3d_reconstruction, mehta2017vnect}
% A potential downside for such a system, however, could be the privacy of the users if cameras are used to determine the user's pose.
% %     -[[ Why mmWave ]]-
% However, mmWave radars solve this issue, as they do not record video, instead, they sense using privacy-preserving point pointclouds.
% 
% %     -[[ Current SOA ]]-
% Systems for HPE using mmWave radar-generated pointclouds already exist; however, these systems all use Deep Learning (DL) to interpret the pointcloud data \cite{an2021mars, an2022fuse, sengupta2020mmpose, junqiao2025diffusion}.
% These systems aim to generate a \textit{skeletal estimation}, a set of key anchor points representative of a human.
% 
% 
% 
% %     -[[ DL issues ]]
% These deep learning approaches have a few downsides, though. 
% They rely on large amounts of qualitative training data and need to be monitored and tested for over- and underfitting.
% Deep learning models also require powerful hardware and consume more electricity than algorithms that perform the same task.
% Lastly, DL models are hard to fine-tune, often requiring retraining.
% %     -[[ Why algorithms ]]-
% An algorithmic solution would solve the reliance on training data, while also providing tuning mechanisms to any users.
% The research into this topic, however, is currently nonexistent.
% Broad integration of HPE-based control systems into IoT environments may not be feasible due to the training and retraining requirements these DL models have.
% Algorithmic HPE methods hold unexplored potential in this regard and could allow for broader adoption of these input systems.
% 
% % Computers are getting more and more integrated into our everyday lives, from smart watches to voice-controlled house assistants, from electric cars to IOT houses.
% % % Therefore, it's important to explore robust, intuitive, and power-efficient interfaces to these systems.