\section{Motivation}
\label{section: introduction - motivation}


%        -------
%        [ GAP ]
%        -------

%     -[[ Why HPE ]]-
Human Pose Estimation (HPE) is a promising new interface to computer systems.
The ability to capture a person's stance and use it as input opens the way to a myriad of control systems.
Most modern HPE solutions use cameras \cite{munaro2014-3d_reconstruction, mehta2017vnect}
A potential downside for such a system, however, could be the privacy of the users if cameras are used to determine the user's pose.
%     -[[ Why mmWave ]]-
However, mmWave radars solve this issue, as they do not record video, instead, they sense using privacy-preserving point pointclouds.

%     -[[ Current SOA ]]-
Systems for HPE using mmWave radar-generated pointclouds already exist, however, these systems all use Deep Learning (DL) to interpret the pointcloud data \cite{an2021mars, an2022fuse, sengupta2020mmpose, junqiao2025diffusion}.
These systems aim to generate a \textit{skeletal estimation}, a set of key anchor points representative of a human.



%     -[[ DL issues ]]
These deep learning approaches have a few downsides, though. 
They rely on large amounts of qualitative training data and need to be monitored and tested for over- and underfitting.
Deep learning models also require powerful hardware and consume more electricity than algorithms that perform the same task.
Lastly, DL models are hard to fine-tune, often requiring retraining.
%     -[[ Why algorithms ]]-
An algorithmic solution would solve the reliance on training data, while also providing tuning mechanisms to any users.
The research into this topic, however, is currently nonexistent.
Broad integration of HPE-based control systems into IoT environments may not be feasible due to the training and retraining requirements these DL models have.
Algorithmic HPE methods hold unexplored potential in this regard and could allow for broader adoption of these input systems.

% Computers are getting more and more integrated into our everyday lives, from smart watches to voice-controlled house assistants, from electric cars to IOT houses.
% Therefore, it's important to explore robust, intuitive, and power-efficient interfaces to these systems.