\section{Research Challenge}
\label{section: introduction - research challenge}

% \ednote{ Why is it hard with mmWave (as opposed to cameras): Sparse pointclouds, high noise percentage\\ This is difficult due to our \textit{high-precision, low-latency} requirements. }
Using mmWave radars over alternative sensors, such as cameras, introduces unique problems.
Unlike cameras, which provide dense, rich, and low-noise data, mmWave radar pointclouds are sparse (20-30 points per frame at 10 frames per second) and noisier.
Some frames are also exceptionally sparse (~5 points) or contain data only of either the right or left half of the user, which makes those frames less valuable.
This sparseness poses a significant challenge when it comes to \textit{real-time}  systems with a \textit{low-latency} requirement.
In order to have a tight response time, the system needs to generate an accurate response in a single frame or a few frames at most.
Since some frames lack data from a specific section of the user, a response needs to be generated with only a single frame of data, or we risk having a noticeable response delay.

% \ednote{ Why is it hard with \textit{algorithms} as opposed to \textit{DL "black boxes"}? \\ No inherent ability to generalise; Researchers will have to \textit{distil} the core features we're looking for. }
The avoidance of DL methods poses another challenge, as they do provide an impressive inherent ability to generalise, even if this is achieved through a \textit{black box}.
The move to an algorithmic solution requires domain-specific prior knowledge to parse the sparse point cloud and infer the actual events that generated that specific input.
This is a difficult challenge when working with high-quality dense data, exacerbated by the noisy nature of mmWave radar point clouds. 
The input data should be fully exploited, using the known physical constraints, in order to observe the underlying events correctly.



% %        -----------
% %        [ PROJECT ]
% %        -----------
% 
% %     -[[ (TECHNICAL) Clear and concise problem statement ]]-
% For this thesis, we will assess the viability of algorithmic interpretation methods for mmWave radar pointclouds for human pose estimation.
% This will take the form of a HPE interface to a musical application, 
% This thesis will create an algorithmic interpretation algorithm for mmWave radar pointclouds.
% This system will measure the angle of the user's arms in real time
% %     -[[ (BRIEF) How and why music ]]- 
% 
% %     -[[ Itemize research challenges ]]- 
% 
% In this thesis, we will explore the possibility of algorithmic interpretation methods of mmWave point clouds.
% We will design a system which can predict the general position of a users arms, into one of three distinct regions, "low", meaning angled at ...., "middle", which is the region of angles ...., and high, given the region of angles ....
% 
% \textbf{Research Challenge}: 
% \begin{itemize}
    % \item 
    % This thesis aims to design an effective algorithmic interpretation method of mmWave radar pointclouds for the purpose of human pose estimation.
    % \item
    % This interpretation method should be explainable, as well as tunable to changing circumstances.
    % \item
    % This interpretation method should hold up, if not exceed early Deep Learning methods for similar purposes.
% \end{itemize}