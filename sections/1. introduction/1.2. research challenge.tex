\section{Research Challenge}
\label{section: introduction - research challenge}


%        -----------
%        [ PROJECT ]
%        -----------

%     -[[ (TECHNICAL) Clear and concise problem statement ]]-
For this thesis, we will assess the viability of algorithmic interpretation methods for mmWave radar pointclouds for human pose estimation.
This will take the form of a HPE interface to a musical application, 
This thesis will create an algorithmic interpretation algorithm for mmWave radar pointclouds.
This system will measure the angle of the user's arms in real time
%     -[[ (BRIEF) How and why music ]]- 

%     -[[ Itemize research challenges ]]- 

In this thesis, we will explore the possibility of algorithmic interpretation methods of mmWave point clouds.
We will design a system which can predict the general position of a users arms, into one of three distinct regions, "low", meaning angled at ...., "middle", which is the region of angles ...., and high, given the region of angles ....

\textbf{Research Challenge}: 
\begin{itemize}
    \item 
    This thesis aims to design an effective algorithmic interpretation method of mmWave radar pointclouds for the purpose of human pose estimation.
    \item
    This interpretation method should be explainable, as well as tunable to changing circumstances.
    \item
    This interpretation method should hold up, if not exceed early Deep Learning methods for similar purposes.
\end{itemize}