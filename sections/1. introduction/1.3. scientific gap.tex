\section{Scientific Gap}
\label{section: introduction - scientific gap}

% All of the current research into HPE with MMWave radars has used Deep Learning (DL) algorithms for data interpretation, and research into algorithmic interpretation is very lacking.
% Deep learning has many strengths, but also many weaknesses, as such, I believe it 

I will talk about the current focus which MMWave HPE research has on Deep Learning, and the scientific gap which exists in non-DL methods.
I'll want to discuss the problems and shortcommings of DL methods.
I also want to talk about the potential benefits which stochastic methods would have: 
% reason 1
Lower power consumption, lower computer resource consumption; 
% reason 2
More room for manual tuning to new environments, as opposed to training on new data (which takes a lot more work); 
% reason 3
The potential for "reasoning" about the good and the bad results, and improving based on that

Due to the aforementioned I think its very useful to explore algoritmic data interpretation systems, since these would be more stable, less resource hungry, and could offer a better basis for any HPE applications.
% Here I'll also lay out my envisoined system, which uses stochastic methods to find the angle of someones arms, while \textbf{briefly} mentioning the example musical application.

