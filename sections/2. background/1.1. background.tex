\section{Background}
\label{section: introduction - background}

\textbf{¿¿¿ Should this be a section or a chapter ???}

This thesis works with a specific sensor, the Millimeter Wave Radar (MMWave Radar), and tries to solve a specific problem, Human Pose Estimation (HPE). 
This section aims to give readers the necessary knowledge on these subjects to be able to read and understand this thesis.
Furthermore, we will discuss some of the current state-of-the-art systems for HPE using an MMWave radar.

% I'll give background on MMWave, HPE, and the current SOA HPE methods used with MMWave.
% I will also discuss some of the properties of MMWave which makes it difficult to handle

\subsection{Millimeter Wave Radar}
\label{sub-section: introduction - background - millimeter wave radar}
Give an explanation of what MMWave radars are, to what extent we can configure them, and what we get out of them (3d pointclouds, at ~10Hz, with between 30-100 points per frame).
Mention the inherent privacy-preserving nature of MMWave due to its not capturing images, and how it fares better in varied light conditions (overexposure/underexposure).
Also, briefly explain how they work internally, using \textit{beam forming antennas} on the receiving and, as well as the large amount of FFT's over the raw signal, to get to pointclouds.

I should mention that the data from MMWave radars is not always temporally stable.
Aka, one frame I might see my left arm, but the next I might see my right.
This is important for section \cref{sub-section: methodology - data filtering - temporal filtering}.

\subsection{Human Pose Estimation}
\label{sub-section: introduction - background - human pose estimation}
Explain what exactly human pose estimation is, and reference some papers discussing it. 
Present the kinect and skeletal estimation, and \textbf{clearly specify} the \textbf{differences} between my \textit{arm location} estimation, and a \textit{full skeletal estimation}.

\subsection{Current State of the Art}
\label{sub-section: introduction - background - current soa}
Briefly mention some \textit{non-mmwave} HPE solutions, those using cameras and those using sensors, as well as their shortcomings (privacy/light situation and clunkiness of wearables, respectively).
Discuss the various SOA MMWave HPE sollutions, their strengths, weaknesses and shortcommings.
Also specify the strengths of Stochastic explainable systems over DL systems (tunable, testable, explainable, significantly lower need for training data), and setup a bridge to the scientific gap.

