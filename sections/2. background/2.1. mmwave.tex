\section{Millimeter Wave Radar}
\label{section: background - millimeter wave radar}

Millimeter Wave Radars (mmWave radar) are a type of sensor which have been discussed for many decades already, with early research dating back to the 1970s and with commercial usage dating back to 1990 in the telecommunication sector, and the automotive industry starting to use it at the turn of the century \cite{menzel2008years}.
In recent years, mmWave radars have been used in a broad number of sensing applications, from human tracking and activity recognition, through to object inspection and classification \cite[Fig.~2]{vanberlo2021millimeter}.

This recent interest in mmWave-based sensing solutions over some of the alternative methods, such as LiDAR and camera-based sensing methods, is due to some of the inherent advantages present in mmWave-based systems.
% vs cameras
Cameras, for example, provide a detailed, high-resolution image of a specific location, but they are susceptible to environmental changes in the light intensity.
mmWave radars do not suffer from this limitation, since they are an active sensor, meaning that they observe the reflection of a signal which has been sent out by the sensor itself.
Furthermore, cameras do not usually include any depth information; they produce a 2-D projection of a scene.
mmWave radars, on the other hand, provide a 3-D point cloud, which does have this range data.
% vs LiDAR
mmWave radars share many of the previously mentioned attributes with other radio-frequency radars, such as LiDAR.
An important difference between LiDAR and mmWave radars, however, is cost.
The high cost of LiDAR makes it an uneconomical option for an everyday interface.
Other radio frequency radars suffer from lower accuracy because the wavelength of the light they send out is longer than that of mmWave radars \cite{yan2025advancements}.
mmWave radars are a cost-effective sensor for this job.


%     -[[ Internal Workings ]]-
To achieve this, mmWave radars transmit a millimeter wave of continuously increasing frequency, and by observing the backscatter, they can infer information about the scene.
% mmWave radars achieve this all in a very clever manner, namely, they send out a wave of constantly increasing frequency and observe the reflection.
Since the frequency is increasing at a constant rate, the distance to the object at which the wave is reflected is proportional to the difference in frequency between the outgoing and incoming signals.
An alternative name for these sensors is thus also a Frequency Modulated Continuous Wave (FMCW) sensor.
The mmWave radar uses a component called a \textit{mixer} to calculate the difference between the ongoing and outgoing signals, which takes two signals as inputs and provides the difference in frequency as well as the difference in phase between those two signals.
This frequency difference holds information about the macro-scale distance, while the phase difference gives information on the micro-scale movements of an object, which is very useful in some specific applications.


The mmWave radar has multiple receivers and multiple senders, which allows the sensor to calculate an elevation and an azimuth for each object.
These different receivers are spaced half a wavelength apart from each other.
This change in position means that the signal will have traveled a different distance, measurable in less than one wavelength, to each of the receivers.
This phase difference can be used to calculate a difference in distance to the object in question, which in turn can be used to calculate the angle of arrival.
The specific layout of Transmitters (tx) and Receivers (rx) on the mmWave radar (see \cref{fig:mmwave_radar_antennas}) means that the mmWave radar can calculate the elevation, azimuth, distance, and Doppler information (which tells us the velocity towards or away from the sensor) of objects in the scene, which is given to the user as a 4-D Pointcloud.
A lot of technical details have been omitted in this explanation, as it is not very important to the workings of the thesis.
For anyone interested in this, the mmWave Radar course by TI\cite{ti2024lessons} is a good starting point.

\begin{figure}[h]
    \begin{minipage}[c]{0.45\linewidth}
        \includegraphics[width=0.95\linewidth]{figures/mmwave/iwr6843isk-top-sideways.png}
        \caption{Top down view of an IWR6843ISK MilliMeter Wave Radar}
        \label{fig:mmwave_radar_top_view}
    \end{minipage}
    \begin{minipage}[c]{0.45\linewidth}
        \includegraphics[width=0.95\linewidth]{figures/mmwave/iwr6842_antenas_up_close.jpeg}
        \caption{Antenna layout for the IWR6843ISK}
        \label{fig:mmwave_radar_antennas}
    \end{minipage}
\end{figure}


\subsection{Data Characteristics}
\label{subsection: background - millimeter wave radar - data characteristics}

%     -[[ Strengths ]]-
The data generated by these mmWave radars has a few specific characteristics.
One, which was already briefly mentioned, is the ability to perceive sub-millimeter distances and movements by using the phase information. 
This ability to measure movements "as small as a fraction of a millimeter" \cite[p.~2]{ti2020fundamentals} is crucial in some applications.
Another unique feature of mmWave radars is their ability to penetrate, and thus look through, certain materials.
These materials include fabrics, such as cotton and leather \cite{meier2020propagation}, but also more solid materials, such as plastic and wood \cite{kapilevich2011fmcw}.
This allows for seamless installation behind otherwise solid structures (such as a plastic case), and can help in the detection of hidden objects and contraband.

%     -[[ Pointclouds ]]
A lot of different data can be gathered from the mmWave radar, which all depends on the application.
For this thesis, however, the focus will be on the pointclouds generated by the mmWave radar, which are generated in most of the applications created with mmWave radars.
%TI provides multiple sample firmwares for its mmWave radars, specialized for different tasks, however, almost all of them generate pointclouds.
These pointclouds generated by the mmWave radar are usually quite sparse, generating around 10 frames each second, usually with no more than 30 points per frame.
Setups that generate higher-density pointclouds are possible, but these often also generate significantly more noise, or try to optimize for one specific factor, such as the Doppler information.
The number of points generated is also dependent on external factors, such as the materials observed, as some are "more visible" and some "less visible" to the mmWave radar, as well as the specific room setup, as this could cause secondary scattering.

%     -[[ Issues ]]-
This means that, overall, the mmWave radar produces quite sparse and noisy pointclouds \cite{brescia2023millinoise}.
The noise often only occurs on one frame at a time and is spatially clustered.
Besides noise, the mmWave radar also suffers from data instability.
Multiple consecutive frames of the same scene can give quite different pointclouds, as some parts of the cloud are \textit{missing} in some frames.
Practically, this means that in one frame, only the right arm is present (in the pointcloud) while in the next frame, only the left arm is present.





