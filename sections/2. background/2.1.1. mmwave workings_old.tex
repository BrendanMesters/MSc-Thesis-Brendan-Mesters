
\subsection{Workings}
\label{subsection: background - millimeter wave radar - workings}
% How do mmWave Radars work

% millimeter waves, and their advantages
Let us first discuss the working of these mmWave radars, as the name already implies, these sensors capture information about the world using electromagnetic radiation in the bandwidth between 30-300 GHz, or in other words, waves with a wavelength between 10-1 millimeter.
mmWave radars have a few distinct advantages because they use electromagnetic waves in this "millimeter wave" bandwidth.
The millimeter waves can, for example, pass through certain fabrics and materials which are opaque to most other sensors, such as cotton and leather \cite{meier2020propagation}, as well as materials such as plastic and wood \cite{kapilevich2011fmcw}.
This makes them ideas for detecting hidden objects, which can be essential in security circumstances.
Furthermore, this ability allows them to be seamlessly integrated into existing facades, creating a more seamless user experience.
The wavelength also allows mmWave radars to detect very small movements, "as small as a fraction of a millimeter" \cite[p.~2]{ti2020fundamentals}, this ability to sense very small changes is another advantage of the chosen bandwidth.

% Basic conceptual working
% mmWave radar sends a signal, it bounces back, and we get a pointcloud "like a bat".
Conceptually, the way in which mmWave radars "sense" the world around them is by sending out a signal (electromagnetic radiation) and observing the return signal, which has bounced off objects in the area to be observed.
In this way, the mmWave radar knows where these objects are, not all that dissimilar to how a bat senses, using echo location.
The distance to the object can be calculated using the time delay between sending the signal and receiving its echo.
By combining this distance with the direction from which the return signal came, the mmWave radar can specify a precise point in 3D space where "something" has been observed.
There are, however, a number of technical difficulties when it comes to making this theoretical system in the real world, the solutions to which will be discussed now.

% How the chirps work (mention FMCW)
% These millimeter waves are sent out as "chirps" by the system, which is why it's also called an FMCW
The mechanism by which an mmWave radar works relies on a few different ideas, which should first be covered.
Firstly, an important part of the mmWave radar is the \textit{chrip}.
mmWave radars sense by sending out a so-called \textit{chirp}, which is a signal of continuously increasing frequency (not unlike the chirps of some birds), which is visualized in \cref{fig:mmwave_chirp}, which is why these radars are also at times called \textit{Frequency Modulated Continuous Wave} (FMCW) radars.
% RX TX IF signal
Since this wave will travel for a certain amount of time before bouncing back to the sensor and being perceived again, it's found that the received return signal (\textbf{RX}) is a time delayed version of the transmitted signal (\textbf{TX}), where the exact time delay is dependent on the distance to the object which is to be observed.
This difference between the currently received signal and the currently transmitted signal's frequencies can be found by feeding both signals into a "mixer", a hardware device that can find the difference between the frequencies of two sine waves, and which will not be discussed in depth in this thesis.
Conceptually, this system is depicted in \cref{fig:mmwave_if_signal}
In this way, the mmWave radar can detect a distance as a time delay (using the known constant speed of light), without having to use any sort of timer, instead using some clever signal processing to get a value which is proportionately equal to the distance.
In reality, however, many such return signals will be generated by a variety of objects in the view window of the mmWave radar, thus, there will not be one singular IF signal, but rather the IF signal will contain multiple "tones", all representative of objects at different distances.
In reality, the system is even still a bit more complicated, but for the purposes of this thesis, this is in-depth enough. 
For a full explanation, please read \cite{introduction_to_mmwave}.


% Side-by-side figures 
\begin{figure}[h]
    \begin{minipage}[c]{0.4\textwidth}
    \includegraphics[width=\linewidth]{figures/mmwave/mmwave_chirp.png}
        \caption{Visualization of a "chirp". \cite{introduction_to_mmwave}}
        \label{fig:mmwave_chirp}
    \end{minipage}
    \hfill
    \begin{minipage}[c]{0.4\textwidth}
        \includegraphics[width=\linewidth]{figures/mmwave/mmwave_IF_signal_cropped.png}
        \caption{The difference between RX and TX represents the distance. \cite{introduction_to_mmwave}}
        \label{fig:mmwave_if_signal}
    \end{minipage}%
\end{figure}


% Beam forming receiver principle
% A principle very similar to that of beam forming antennas is used to interpret this data.
In order to determine where the perceived object is, the mmWave radar needs to know the angle or arrival of the return signal.
For this, the mmWave uses an array of receiver antennas that are spaced half a wavelength apart, as shown in \cref{fig:mmwave_radar_antennas}. 
By measuring the difference in phase of the returned signal at each of the receiver antennas, the mmWave radar can infer from what lateral direction the signal came.
This is possible because, as the signal travels, the electromagnetic wave propagates, and its exact phase changes.
This phase can be detected by the mmWave radar, by using the previously mentioned "mixer", the specifics of which are not important to this thesis.
By comparing the phase of these two incoming signals, an angle can be derived, since the receivers are separated by a known distance, namely, half the wavelength ($\lambda$).
In a similar way, the mmWave radar can determine the elevation of an object by using the three transmitting antennas, since one of them is located with a slight vertical offset to the other two.

\begin{figure}[h]
    \begin{minipage}[c]{0.45\linewidth}
        \includegraphics[width=0.95\linewidth]{figures/mmwave/iwr6843isk-top-sideways.png}
        \caption{Top down view of an IWR6843ISK MilliMeter Wave Radar}
        \label{fig:mmwave_radar_top_view}
    \end{minipage}
    \begin{minipage}[c]{0.45\linewidth}
        \includegraphics[width=0.95\linewidth]{figures/mmwave/WhatsApp Image 2025-10-28 at 15.59.05.jpeg}
        \caption{Antenna layout for the IWR6843ISK}
        \label{fig:mmwave_radar_antennas}
    \end{minipage}
\end{figure}

% 4 FFTs to get a pointcloud
% With a lot of FFTs, the data gets transformed from … to … to … to pointclouds.
The raw data being read by the mmWave radar is a complex waveform, comprised of many combined sinusoids, whose frequencies and phases are dependent on the objects being observed in the view of the radar.
The radar produces such a stream for each of its receiving antennas.
This data needs to be transformed to get useful information out of it, which is done by applying the \textit{Fourier Transformation} multiple times.
First, a Fourier Transform is used to convert this data from the \textit{time domain} to the \textit{frequency domain}, which should show the different distances at which objects have been observed, since the frequencies observed were dependent on the travel time of the mmWave.
Now this is done for a number of chirps, such that you can construct a matrix with the \textit{frequency domain} on one axis, and the \textit{chirp number} on the other, and another Fourier transform is considered, with regards to the \textit{chirp axis}, resulting in a \textit{range-velocity} matrix.
Lastly, the data from the different antennas is combined, giving a \textit{range-velocity-antenna} matrix. 
Suppose a Fourier transform is applied over this matrix, w.r.t. the antennas. 
In that case, the resulting matrix will be one of \textit{range-velocity-angle}, which can be interpreted as a point cloud.

\textbf{SIMPLIFY THE PART ABOVE}
