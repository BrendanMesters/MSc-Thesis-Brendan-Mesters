\subsection{pointclouds}
\label{subsection: background - millimeter wave radar - pointclouds}
% Describe the pointclouds they generate, and some of their attributes (Doppler)

% What is a pointcloud
The output generated by the mmWave radar can vary somewhat, based on the firmware 
%(discussed in \cref{subsection: background - millimeter wave radar - configuration})
However, all firmware options do output \textit{point clouds}, in one representation or another.
In the case of mmWave radars, pointclouds are a collection of 4-dimensional vectors.
These vectors each have the 3 spatial dimensions, X, Y, and Z.
% Doppler
The last dimension is called the "Doppler" data, which is what, in the previous section, was called the "velocity".
This is done, as an object moving towards the sensor will increase the frequency of the millimeter wave by way of the Doppler effect, and likewise it will decrease the frequency if it's moving away.
In this way, the manner in which the velocity towards, and away from, the sensor is measured is by using the Doppler frequency shift that occurred.
This doppler information is quite unique to mmWave radars, since it provides a measure of how fast an object was moving at the moment a frame is captures, where most systems can only provide an estimation of the movement between frames.

% Characteristics of mmWave radar point clouds
There are many systems that generate pointclouds as an output, for example, lidar and RGBD cameras. 
It's thus important to look at the characteristics of pointclouds generated by mmWave radars.
One important way in which mmWave radars differ from these other systems is the high resolution that they can achieve, due to the wavelength used.
This allows mmWave radars to be used for various tasks that similar systems can't perform, such as the monitoring of vital signs, such as the heartbeat and respiratory rate \cite{wang2022heartprint, wang2023here} or systems which can reproduce audio, based on the vibrations of objects \cite{li2025acoustic, Lin2024highquality, Li2022mmphone}.
One disadvantage of mmWave radars, however, is that they produce very sparse pointclouds.
The experiments during this thesis have found that the mmWave Radar produces between 20-100 points per frame, at a rate of roughly 10 frames per second. 
This poses a significant challenge for any interpretation system, certainly those that try to interpret the mmWave radar point clouds in real time, with latency constraints.



% caveats and issues
During the design of IAmMuse, some issues with the mmWave radar point clouds were found, namely issues with noise and data stability.
Firstly, the noise, during this thesis, it was found that the mmWave radar is prone to producing high levels of noise due to reflections, the prevalence of this type of noise is also dependent on the specific environment in which the mmWave radar is used.
Besides this relatively regular environmental noise, noise spikes were also encountered semi-regularly during the use of the mmWave radar.
These noise spikes were often only present for one frame at a time, after which they disappeared again.
Another issue that was encountered was the distribution of points, namely, often a single frame would either hold points belonging to the right side of the user or points belonging to the left side of the user.
It's not clear if these issues were inherent to the chip we used, caused by misconfiguration, or caused by damage to the chip.


% \subsection{configuration}
% \label{subsection: background - millimeter wave radar - configuration}
% Texas Instruments provides several firmware setups \cite{ti_mmwave_firmware} for various of its mmWave radar chips, each implements separate interpretation methods over the raw data which is generated by the FMCW.

% Similar to lidar, mmWave radars are an active sensor, meaning that they do not rely on an external source of light, as they measure the reflection of a signal they send out themselves. 
% This means that they will work in any light situation, as opposed to cameras, which won't function properly if there is either too little or too much light.




% Give an explanation of what MMWave radars are, to what extent we can configure them, and what we get out of them (3d pointclouds, at ~10Hz, with between 30-100 points per frame).
% Mention the inherent privacy-preserving nature of MMWave due to its not capturing images, and how it fares better in varied light conditions (overexposure/underexposure).
% Also, briefly explain how they work internally, using \textit{beam forming antennas} on the receiving and, as well as the large amount of FFTs over the raw signal, to get to pointclouds.
% 

