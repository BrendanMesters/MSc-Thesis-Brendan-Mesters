\section{Human Pose Estimation}
\label{section: background - human pose estimation}

%     -[[ What is HPE ]]-
Human Pose Estimation (\textbf{HPE}) is a broad term covering systems that estimate a human's pose, usually in the form of a set of 3D coordinates.
A well-known example of an HPE system is the Microsoft Kinect system, a camera-based movement tracker originally introduced for the Xbox game console.
Though HPE can cover a broad number of applications, many of them aim to construct a \textit{skeletal estimation}, a set of essential keypoints to represent a human skeleton.
Microsoft Kinect, in particular, generates a skeletal estimate composed of 25 points, as depicted in \cref{fig:kinect_skeletal_estimate}, and is used as a source of ground truth for several other HPE systems.

\begin{figure}[h]
    \centering
    \includegraphics[width=0.5\linewidth]{figures/misc/The-skeleton-of-the-Kinect-v2-19.png}
    \caption{Kinect's 25-point skeletal estimation \cite{guffanti2020accuracy}}
    \label{fig:kinect_skeletal_estimate}
\end{figure}

%     -[[ Why do we want HPE ]]-
HPE methods have become increasingly prevalent as computer systems have become more integrated into our everyday environment, providing a non-invasive way to interface with them.
These HPE systems can be used for a variety of purposes; the Kinect, for example, was originally built for gaming.
Other fields have also started using HPE methods.
Activity recognition, through kinematic skeletal estimations, can enhance existing security systems, allowing them to detect security threats autonomously.
Action recognition and tracking methods can control IoT systems, and HPE systems have even seen use in automated sign language recognition, assisted living spaces, and driver assistance systems \cite{munea2020progress}.


%     -[[ What methods exit ]]-
HPE systems exist for various input modalities.
Vision-based methods \cite{Lan2023visionbased} often use normal cameras or RGB-D cameras to construct pose estimations. 
Wearable-based methods \cite{moniruzzaman2023wearable} rely on sensors worn by the user to construct their estimations.
Most of these methods, however, are built with a deep learning framework at the core in order to transform the sensor input into a prediction.
