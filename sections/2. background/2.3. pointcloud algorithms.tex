\section{algorithmic pointcloud methods?}
\label{section: background - algorithmic pointcloud methods}

This thesis is, for a large part, based on a comparison between deep learning methods and algorithmic methods, both of which have advantages and disadvantages.
A large advantage of algorithmic systems is that they can be reasoned about and can be configured in a clear manner with the use of input parameters.
For pointcloud analysis, some great and concise clustering algorithms have been created, such as DBSCAN \cite{ester1996dbscan}, and some problems are being successfully tackled by both DL systems, as well as algorithmic systems, such as geometric labeling of pointcloud data \cite{poux2019voxel}, or audio reconstruction using mmWave radars \cite{li2025acoustic}.



% A part of the different algorithmic methods to handle large swaths of high-dimensional data, specifically for clustering.
% This has gotten a lot of recent attention due to Machine Learning implementations often transforming data into a high-dimensional space.

% \textbf{DBSCAN signal processing algorithm} \cite{ester1996dbscan}\\
% \textbf{SOA acoustic reconstruction system} \cite{li2025acoustic}