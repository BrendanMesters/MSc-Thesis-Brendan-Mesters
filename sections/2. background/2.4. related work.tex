\section{Related Work}
\label{section: background - related work}

% [[ Broad background ]]
% - Deep Learning
% - HPE
% - Pointclouds


% [[ Narrowing ]]
% - mmWave
% - pointcloud DL systems
% - algorithmic pointcloud processing (DBSCAN, tracking, etc)

% [[ Specific ]]
% - mmWave HPE SOA
% - simpler models (MARS) and more complex models (???)
% - no existing algorithmic mmwave hpe methods.


%                  ---------
%                  [ BROAD ]
%                  ---------

%     -[[ General MMW Systems ]]-
A lot of new research uses mmWave radar technology.
Research into vital sign monitoring \cite{wang2022heartprint,wang2023here} uses it for its ability to sense sub-millimeter-sized movements.
% It is used for its ability to sense sub-millimeter-sized movements, in the medical field, to monitor vital signs \cite{wang2022heartprint,wang2023here}.
This ability to capture small movements can also be used to capture sound through otherwise soundproof barriers \cite{lin2024highquality, li2025acoustic}.
Due to mmWave radars' privacy-preserving nature, it is also seeing a lot of use in sensing systems for public spaces.
People counting and tracking systems \cite{}, activity detection systems \cite{sen2024multi-user}, and object detection and classification systems \cite{} are the most prevalent examples of this.
% mmWave radar is also broadly being used in object detection and classification algorithms \cite{} as well as 

%     -[[ HPE Methods/Hardware ]]-
% mmWave radar is one method used to implement HPE systems, but the field uses a variety of sensors.
% Many HPE systems rely on cameras, with or without a depth channel, 

%     -[[ Music Generation ]]-
New technologies have always been put to use in new musical instruments.
The electronic piano/synthesizer is one of the first and most well-known of these technological inventions, which became an instrument.
The themerim is another well-known musical instrument that originated in this field, among many other lesser-known examples \cite{mcglynn2014interaction}.
These new instruments brought with them a new wave of musical intrigue \cite{otto1968history} and musical expression.
%     -[[ MMW Music ]]-
mmWave radar has already been used for music generation, either through gestures \cite{juneja2024wavetune}, or by movement close to the sensor \cite{bernado2017o_soli_mio}, though the user freedom in these systems is limited.
% Both don't really provides musical instrument control. 
% They either let you pick between samples, or bind some variable of a musical script to some variable of the mmwave sensor.



%                  ------------
%                  [ NARROWER ]
%                  ------------

%     -[[ Pointcloud Algorithms ]]-
% Pointclouds come up with many different technologies and in many different systems, therefore, a good number
This thesis is, for a large part, based on a comparison between deep learning methods and algorithmic methods, both of which have advantages and disadvantages.
A major advantage of algorithmic systems is that they can be reasoned about and configured clearly using input parameters.
For pointcloud analysis, some great and concise clustering algorithms have been created, such as DBSCAN \cite{ester1996dbscan}, and some problems are being successfully tackled by both DL systems, as well as algorithmic systems, such as geometric labeling of pointcloud data \cite{poux2019voxel}, or audio reconstruction using mmWave radars \cite{li2025acoustic}.




%     -[[ MMW HPE ]]-


%                  -------------
%                  [ NARROWEST ]
%                  -------------

%     -[[ Does not exist -> IAmMuse ]]-




\brendanLines
Generic point cloud algorithms

Early and more recent HPE methods (camera, wearable, wifi, mmwave)

DL vs algorithmic solutions

algorithmic mmWave/pointcloud solutions: DBSCAN, etc

HPE with mmwave (the DL systems, including mars)

Algorithmic HPE for mmWave pointclouds \textbf{does not yet exist}

\subsection{Current State of the Art}
\label{sub-section: background - related work - current soa}
Briefly mention some \textit{non-mmwave} HPE solutions, those using cameras and those using sensors, as well as their shortcomings (privacy/light situation and clunkiness of wearables, respectively).
Discuss the various SOA MMWave HPE sollutions, their strengths, weaknesses and shortcommings.
Also specify the strengths of Stochastic explainable systems over DL systems (tunable, testable, explainable, significantly lower need for training data), and setup a bridge to the scientific gap.
