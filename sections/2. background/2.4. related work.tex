\section{Related Work}
\label{section: background - related work}

%                  -----------------
%                  [ NEW STRUCTURE ]
%                  -----------------

For the related works, two different types of work will be considered: those related to innovative musical expression and those pertaining to mmWave technology.
This thesis aims to use the sensing capabilities of an mmWave radar to create an instrument that is controlled by the user's arm movements.
Similar \textit{full body} musical interfaces already exist.
One such system is Somi-1 \cite{somi-1}, which uses motion sensors to control music with dynamic movement.
MiMu gloves \cite{mimu} is another similar system that can only attach to the user's hands, but can also measure the bend of the user's fingers, in addition to motion.
There are also two existing systems that use mmWave radars for musical expression:
% There already exist two mmWave-based musical systems, but they have a different use case than IAmMuse.
\textit{WaveTune} \cite{juneja2024wavetune}, which uses gestures to select a combination of prebuild beat-patterns; 
and \textit{O Soli Mio} \cite{bernado2017o_soli_mio}, which allows you to control specific musical effects, or switch between different samples, by performing gestures above a sensor.
IAmMuse, on the other hand, works more similarly to a traditional instrument, allowing the user to generate specific musical notes using their arm movement.

There exists a wide breadth of mmWave radar data interpretation systems, with either a Deep Learning (DL) or an algorithmic approach.


\ednote{Algorithmic applications; and why we want one}


\ednote{Existing mmWave HPE, which all use DL}

\ednote{DL downsides, advantages and tradeoffs of IAmMuse (simpler, cheaper, no training) vs (simpler problem space, "arm position estimation" not HPE)}


% Similar concept musical experiences: Mimu \cite{mimu} and Somi-1 \cite{somi-1}.



\brendanLines

%     -[[ MUSIC ]]-
% Establish broad context for the engineering challenge.
% Mention that new technology often drives new forms of expression.
% Transition to new technology: mmWave
New technologies often lead to new forms of expression.
The electronic piano/synthesizer is one of the first and most well-known examples of an instrument made possible by the technological innovations at the time.
The theremin is another well-known musical instrument that emerged from a new invention, among many other lesser-known examples \cite{mcglynn2014interaction}.
These new instruments brought with them a new wave of musical intrigue \cite{otto1968history} and musical expression.
%     -[[ MMW Music ]]-
mmWave radars are another such system that can be used for new innovative ways of expression, and have already been used for this purpose in a limited manner.
The currently existing tools for mmWave music generation work either through gestures \cite{juneja2024wavetune}, or by movement close to the sensor \cite{bernado2017o_soli_mio}.
% though these systems limit the user's freedom of expression with their input modality.
These systems only allow the user to change between pre-existing music tracks or to affect some variables in an audio effect, thus limiting the scope of musical expression.


%     -[[ mmWave SOA (broad) ]]-
% Give the context of what is happening with mmWave
% Mention the various systems its being used in
Numerous innovative research papers use mmWave radar technology.
Research into vital sign monitoring \cite{wang2022heartprint,wang2023here} and acoustic reconstruction \cite{li2025acoustic, lin2024highquality} both use mmWave radar for its ability to sense sub-millimeter-sized movements.
% Research into vital sign monitoring \cite{wang2022heartprint,wang2023here} uses mmWave radar for its ability to sense sub-millimeter-sized movements.
% This ability to capture small movements can also be used to capture sound through otherwise soundproof barriers \cite{lin2024highquality, li2025acoustic}.
Systems designed for sensing in public spaces use mmWave radar for its privacy-preserving nature
% Due to mmWave radars' privacy-preserving nature, it is also seeing a lot of use in sensing systems for public spaces.
These systems range from people counting and tracking systems \cite{zhao2019tracking, vaidya2024exploiting}, activity and gesture recognition systems \cite{sen2024multi-user, palipana2021pantomime}, through to object detection and classification systems \cite{qi2016pointnet, qi2017pointnet++, guan2020through}.

%     -[[ Algorithms vs Deep Learning ]]-
% Open the discussion of Algorithms vs DL
% Give the pros and cons of both sides
% Show some algorithmic SOA systems.
Most research into pointcloud interpretation uses Deep Learning over algorithmic methods, in spite of both having distinct advantages and disadvantages.
% This thesis is, for a large part, based on a comparison between deep learning methods and algorithmic methods, both of which have advantages and disadvantages.
A significant advantage of algorithmic systems is that they can be explicitly reasoned about and allow for precise configuration using well-specified input parameters.
% allow for explicit reasoning and concisely configure them using well-specified input parameters.
Conversely, Deep Learning models offer a remarkable ability to generalize, but suffer from having opaque inner workings.
%be reasoned about and configured clearly using input parameters.
Robust and established clustering algorithms already exist for pointcloud analysis, such as DBSCAN \cite{ester1996dbscan}.
Other problems are successfully tackled by both DL systems, as well as algorithmic systems, such as geometric labeling of pointcloud data \cite{poux2019voxel}, or audio reconstruction using mmWave radars \cite{li2025acoustic}.
%For pointcloud analysis, some great and concise clustering algorithms have been created, such as DBSCAN \cite{ester1996dbscan}, and some problems are being successfully tackled by both DL systems, as well as algorithmic systems, such as geometric labeling of pointcloud data \cite{poux2019voxel}, or audio reconstruction using mmWave radars \cite{li2025acoustic}.


%     -[[ mmWave HPE | the gap]]
% Discuss the current SOA in mmWave HPE.
% Introduce the GAP of algorithmic HPE.
Currently, there is a lack of research into algorithmic HPE methods using mmWave radars.
The vast majority of research in the field focuses either on deep learning interpretation models \cite{mars2021github, junqiao2025diffusion, sengupta2020mmpose} or on constructing data sets for those deep learning models \cite{brescia2023millinoise, su2025high_fidelity, cui2023milipoint}.
Algorithmic interpretation methods for mmWave data can match or even exceed the results of similar Deep Learning systems, thus it is worthwhile to explore algorithmic methods for mmWave HPE.










% \brendanLines
% [[ Broad background ]]
% - Deep Learning
% - HPE
% - Pointclouds


% [[ Narrowing ]]
% - mmWave
% - pointcloud DL systems
% - algorithmic pointcloud processing (DBSCAN, tracking, etc)

% [[ Specific ]]
% - mmWave HPE SOA
% - simpler models (MARS) and more complex models (???)
% - no existing algorithmic mmwave hpe methods.


%                  ---------
%                  [ BROAD ]
%                  ---------

%     -[[ General MMW Systems ]]-
% A lot of new research uses mmWave radar technology.
% Research into vital sign monitoring \cite{wang2022heartprint,wang2023here} and acoustic reconstruction \cite{li2025acoustic, lin2024highquality} both use mmWave radar for its ability to sense sub-millimeter-sized movements.
% % Research into vital sign monitoring \cite{wang2022heartprint,wang2023here} uses mmWave radar for its ability to sense sub-millimeter-sized movements.
% % This ability to capture small movements can also be used to capture sound through otherwise soundproof barriers \cite{lin2024highquality, li2025acoustic}.
% Due to mmWave radars' privacy-preserving nature, it is also seeing a lot of use in sensing systems for public spaces.
% People counting and tracking systems \cite{zhao2019tracking, vaidya2024exploiting}, activity and gesture recognition systems \cite{sen2024multi-user, palipana2021pantomime}, and object detection and classification systems \cite{qi2016pointnet, qi2017pointnet++, guan2020through} are the most prevalent examples of this.
% % mmWave radar is also broadly being used in object detection and classification algorithms \cite{} as well as 
% 
% %     -[[ HPE Methods/Hardware ]]-
% % mmWave radar is one method used to implement HPE systems, but the field uses a variety of sensors.
% % Many HPE systems rely on cameras, with or without a depth channel, 
% 
% %     -[[ Music Generation ]]-
% New technologies have always been put to use in new musical instruments.
% The electronic piano/synthesizer is one of the first and most well-known of these technological inventions, which became an instrument.
% The themerim is another well-known musical instrument that originated in this field, among many other lesser-known examples \cite{mcglynn2014interaction}.
% These new instruments brought with them a new wave of musical intrigue \cite{otto1968history} and musical expression.
% %     -[[ MMW Music ]]-
% mmWave radar has already been used for music generation, either through gestures \cite{juneja2024wavetune}, or by movement close to the sensor \cite{bernado2017o_soli_mio}, though the user freedom in these systems is limited.
% % Both don't really provide musical instrument control. 
% % They either let you pick between samples, or bind some variable of a musical script to some variable of the mmwave sensor.
% 
% 
% 
% %                  ------------
% %                  [ NARROWER ]
% %                  ------------
% 
% %     -[[ Pointcloud Algorithms ]]-
% % Pointclouds come up with many different technologies and in many different systems, therefore, a good number
% This thesis is, for a large part, based on a comparison between deep learning methods and algorithmic methods, both of which have advantages and disadvantages.
% A major advantage of algorithmic systems is that they can be reasoned about and configured clearly using input parameters.
% Robust and established clustering algorithms already exist for pointcloud analysis, such as DBSCAN \cite{ester1996dbscan}.
% Other problems are successfully tackled by both DL systems, as well as algorithmic systems, such as geometric labeling of pointcloud data \cite{poux2019voxel}, or audio reconstruction using mmWave radars \cite{li2025acoustic}.
% 
% 
% 
% 
% %     -[[ MMW HPE ]]-
% Most 
% 
% %                  -------------
% %                  [ NARROWEST ]
% %                  -------------
% 
% %     -[[ Does not exist -> IAmMuse ]]-









% \brendanLines
% Generic point cloud algorithms
% 
% Early and more recent HPE methods (camera, wearable, wifi, mmwave)
% 
% DL vs algorithmic solutions
% 
% algorithmic mmWave/pointcloud solutions: DBSCAN, etc
% 
% HPE with mmwave (the DL systems, including mars)
% 
% Algorithmic HPE for mmWave pointclouds \textbf{does not yet exist}
% 
% \subsection{Current State of the Art}
% \label{sub-section: background - related work - current soa}
% Briefly mention some \textit{non-mmwave} HPE solutions, those using cameras and those using sensors, as well as their shortcomings (privacy/light situation and clunkiness of wearables, respectively).
% Discuss the various SOA MMWave HPE sollutions, their strengths, weaknesses and shortcommings.
% Also specify the strengths of Stochastic explainable systems over DL systems (tunable, testable, explainable, significantly lower need for training data), and setup a bridge to the scientific gap.
