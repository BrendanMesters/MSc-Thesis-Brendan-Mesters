
\section{Testing Setup}
\label{section: setup baseline dataset - testing setup}

%     -[[ Hardware Setup ]]-
% - What exact chip do we use, what exact Kinect
% - What firmware do we run, and why
% - refer to the appendix with more in-depth info
% - 
In this thesis, a few pieces of Hardware were used.
% FMCW
For the mmWave Radar, the IWR6843ISK chip by Texas Instruments was used.
This chip needs to run a specific firmware to work, TI provides a set of ready-made firmware, which differ slightly in what they provide to the end user.
The main aspects that are important to this thesis are the number of points generated, the noise percentage in those points, and the "usefulness" of each point.
For this thesis, a "useful" point can tell you a lot about the position of a (semi-static) arm.
These criteria were met by the \textit{Mobile Tracker} firmware \cite{ti_mmwave_firmware}, for this reason, and the fact that the firmware aligned well with an interpretation method which was being explored at the time, this became the firmware of choice.
For more detail, read Appendix \ref{appendix: configuration in detail}
% Kinect
Next to the mmWave Radar, a Kinect 2 was used to collect ground truth data.
The skeletal estimation was generated through a code base published by Kharche on GitHub \cite{kharche2019skeletal}, which creates a smooth skeletal estimation and saves it, along with timestamps, to a CSV file.

%     -[[ Physical setup ]]-
% - What does the room look like
%   - Where was the sensor
%   - Where was the user
% - What effect could the room setup have
% - 
The experiments were performed in a large open room (7 by 10 meters), the mmWave radar was mounted at a height of 125 CM and was tilted very slightly (~5\degree) upwards.
The Kinect was mounted at a height of 175 CM and angled somewhat downwards.
A screen was also present, on which the user could see their currently predicted zones, and the incoming pointclouds.
The user was standing at a distance of 2 meters from the sensors.
Appendix \ref{appendix: experimental setup} shows the experimental setup in detail.
The room being large and open meant that back wall reflections were spatially very distinct from the user.
The back wall was also broken up with open shelving, which should've scattered the incoming millimeter waves, as opposed to reflecting them.
The large screen present was angled upwards, and would've thus not reflected the signal back to the mmWave sensor.
These factors meant that the mmWave radar produced pointclouds with relatively little noise in this room, which in turn meant that the noise reduction algorithms did not have to be tuned up as much.



%     -[[ Testing procedures]]-
% - How was the user informed
% - What were they asked to do
% - How did we provide feedback
% - Refer to the appendix's giving more in-depth info
% - 
Users were informed of how the system worked and what they were expected to do through an explanatory PDF (see Appendix \ref{appendix: user explanation}).
The specific actions the users performed will be discussed in \cref{section: setup baseline dataset - dataset}.
After each recording, the researcher provided some small feedback to the user about how to best use the system. 
Commonly given advice was to wave their hands, or do "jazz hands", as well as informing the user where the optimal "zone positions" are (for low, mid, high).
A more detailed overlook of the advice which was given, along with notes of the user performance, can be found in Appendix \ref{appendix: experiment notes}.


%     -[[ ]]-
% - 
% - 
% - 
%     -[[ ]]-

% Double separation line and spacing
% \vspace{2em}
% \hrule
% \vspace{0.2em}
% \hrule
% \vspace{2em}

%     [[ Hardware setup (chip, firmware, config ]]
% Describe the chip we used, the config we used on that chip, the reason for that choice, the specific config used.
% Also, describe the method for ground truth collection.

% The mmWave radar used for this thesis was the IWR6843ISK by Texas Instruments.
% The firmware which is used on this chip is the \textit{Mobile Tracker} firmware \cite{ti_mmwave_firmware}, in part chosen due to it producing a reasonable number of relatively representative points, and in part due to legacy reasons, which are further discussed in Appendix \ref{appendix: configuration in detail}.
% Next to this, a Microsoft Kinect 2 was used to collect the ground truth.
% To get a skeletal estimation from the Kinect, a code base by \citeauthor{kharche2019skeletal}\cite{kharche2019skeletal} was used.
% This code base could record the skeletal estimation produced by the Kinect to a CSV file, with timestamps for each frame, making it a good choice for the ground truth.


%     [[ Hardware setup p2. ]]
% Discuss the specific firmware used, what the advantages and disadvantages are, and what other options we might have had.
% -- [ SKIP ] -- 

%     [[ Environment setup (mounting, space) ]]
% Describe the room in which the tests were held, describe the positions for the various sensors, as well as the position of the user
% Use photos, specify measurements/distances

% [[ Testing procedure ]]
% Describe the testing procedure.
% Describe the types of data that were collected and why,
% Describe the way in which the user was guided.

% Explain the type of room we took measurements in and why, also describe the sensors we use (fmcw + kinect)

% Describe the test setup specifically, use photos and specify measurements (user stood at a distance of 2m)

% Explain the testing procedure, what did we tell the users?