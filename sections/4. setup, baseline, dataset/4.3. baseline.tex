
\section{Baseline Analysis}
\label{section: setup baseline dataset - baseline analysis}

%     -[[ Introduction ]]
% - Why do we want to compare
% - What are some of the criteria a comparison system must fulfill
%   - Simple
%   - "comparable"
%   - Easy to get running.
% - 
IAmMuse should also be compared to a similar Deep Learning model to see how well this algorithmic approach holds up to these more conventional systems, for this type of problem.
There are a few criteria that an interpretation model should fulfill in order to be considered for use in this thesis.
% early model
One important consideration is that the model should be a relatively early attempt at HPE interpretation through DL methods.
Since IAmMuse is the first attempt at algorithmic data interpretation for HPE, a fair comparison would put it up against an early DL system.
% Same data
The DL system should be able to run with the data that this thesis already has, so it needs to be able to take a Kinect skeletal estimation as a ground truth, and to take the pointclouds collected for this thesis as training data.
% Easy to get going
Lastly, for practical reasons, it should be relatively easy to get working, due to the general time constraints on a master's thesis.


%     -[[ MARS ]]
% - How does MARS fulfill the criteria
% - Short synopsis of MARS system (e.g. Skeletal estimation through deep learning)
% - Small notice of "training" versions
% - 
With these considerations, the MARS system \cite{an2021mars} was chosen.
This is a relatively early system when it comes to mmWave radar interpretation for HPE, making it a fair comparison against IAmMuse.
MARS was also trained by using the pointcloud output of a previous model of mmWave Radar from TI (the IWR1443) to the one that was used for this thesis (IWR6843).
Next to that, the ground truth used in the MARS paper was also collected with a Kinect 2.
Lastly, this model has already been used in our group, which meant that it was relatively easy to get it up and running.
The exact method by which MARS produces a skeletal estimation is beyond the scope of this thesis, and can be found in the paper itself, however, it's still important to go over some basics.
The MARS model has to \textit{train} on a set of data, the predictions it will make are based on this training.
The MARS system also only takes into account a single frame at a time, therefore, it has no consensus of temporality.

%     -[[ MARS variations ]]
% - What "MARS versions"/trainings will we use
% - Why were these chosen
% - What are the expectations for each one?
% - 
As was already mentioned, MARS needs to be trained on a dataset to work. 
This thesis will compare IAmMuse with multiple different trained versions of the MARS model, which we will specify here.

\begin{table}[!htb]
    \centering
    
    \resizebox{\linewidth}{!}{%
        \begin{tabular}{|p{0.30\linewidth}|p{0.5\linewidth}|p{0.5\linewidth}|}
        \hline
             \textbf{Model Name} & \textbf{Training Data} & \textbf{Testing Data}  \\
             \hline\hline
             MARS baseline  
                & Calibration data of \textit{one specific recording}.
                & Usage data of \textit{the same specific recording}.
             \\ \hline
             MARS paper 
                & MARS data set.
                & Usage data of all recordings.
             \\ \hline
             MARS full calib 
                & Calibration data of all recordings.
                & Usage data of all recordings.
             \\ \hline
             MARS partial standard 
                & Usage data of 2/3 standard recordings. 
                & Usage data of the other 1/3 standard recordings.
             \\ \hline
             MARS standard to free 
                & Usage data of all standard recordings.
                & Usage data of all free-play recordings.
             \\ \hline
        \end{tabular}
    }
    \caption{The training and testing data for the various MARS models considered}
    \label{figure: MARS train test}
    
\end{table}


%     -[[ ]]
% - 
% - 
% - 

%     -[[ ]]
% - 
% - 
% - 
%     -[[ ]]


% Double separation line and spacing
\vspace{2em}
\hrule
\vspace{0.2em}
\hrule
\vspace{2em}


% Compare with a DL model
Bring up the point of comparison, 
Mention that we need a "baseline" for comparison.
A baseline should be a deep learning model, since all current SOA systems are DL.
The baseline should be a simple/early model, as IAmMuse is also a first exploration
Propose MARS as a good model for comparison.

Discuss MARS, the strengths and weaknesses according to its paper (single frame only, no temporality).
Describe that we will want to train it on our own data.
Briefly discuss the different training methods (single calib, all calib, MARS paper, standard to free, maybe partial standard).

For this, use a table with the different train test data sets.
Also, argue for each of these options, why they'd be interesting to look at.
Give a small "alleup" for evaluation.

% Argue why MARS was used as a comparative model

% Discuss the training method of MARS (epocs, etc)

% Briefly describe the different training methods which will be discussed.
















% -- [ OLD ] --

% Describe the simplest training (trained on calib data)

% Describe some other "more advantageous" trainings we could do for Mars (all calib data, the MARS model itself, a fully representative section of the data).



