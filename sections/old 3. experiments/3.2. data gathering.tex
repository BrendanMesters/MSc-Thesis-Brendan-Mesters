\section{Data Gathering}
\label{section: experiments - data gathering}
I want to describe the process of data gathering, how I did \textbf{3 standardized runs}, where I gave tips in between, the type of tips I gave (waving your hand, not your arm, to the camera, or doing jazz hands; Where the zones where the zones got the best data).

I want to discuss the formats in which the data was gathered, and why that choice was made.


\subsection{Formal Data Explanation}
\label{sub-section: experiments - data gathering - formal data explanation}

In order to understand the results, it's important to know exactly how these results were achieved.
During the data recording, the users were presented with the musical output of IAmMuse. 
Each recording has a number of frames (between \textbf{FIND DATA RANGE AND MEDIAN} frames), which were used to \textit{calibrate} IAmMuse to the specific user, the rest of the frames are the frames that IAmMuse used to generate its output and will thus be used for its evaluation.
These two parts of the recording will be referred to as the \textbf{calibration data} and the \textbf{usage data} respectively, and an exhaustive list of these frame counts can be found in \cref{appendix: recording frame distribution}.

Each specific user has \textit{three recordings} where they perform a \textbf{standardized movement set}.
For each specific user, there are also \textit{between one and four} \textbf{free play} recordings (users were encouraged, but not mandated to use the system to their own accord).
If a \textit{specific recording} is referenced, it will be done as follows: 
\begin{itemize}
\item "\textbf{standard movement set} recording from \textbf{user 07} number \textbf{2 out of 3}".
\item or "\textbf{free play} recording from \textbf{user 52} number \textbf{1 out of 1}
\end{itemize}

The MARS system requires training data in order to be able to evaluate anything, the specific selection of this training data is very important.
For this evaluation, the MARS system has been trained and tested in a few distinct ways.\\
% MARS pre-trained
\textbf{First}, we used the MARS system with the trained model provided by \citeauthor{an2021mars} in the paper of MARS\cite{an2021mars}. 
This model was then evaluated on the \textit{usage data} of \textbf{all recordings}.
This evaluation will be referred to as "\textit{MARS pre-trained}".\\
% MARS calibration trained
\textbf{Second}, the MARS model is trained (in accordance with the specifications on their GitHub page \cite{mars_github}) with the \textit{calibration data} from \textbf{all recordings}.
80\% of this data will be used as a \textit{training set} and 20\% of this data will be used as a \textit{validation set}.
This model was then evaluated (tested) on the \textit{usage data} of \textbf{all recordings}.
This evaluation will be referred to as "\textit{MARS calibration data trained}".\\
% MARS full trained
\textbf{Third}, the MARS model is trained (in accordance with the specifications on their GitHub page \cite{mars_github}) with \textit{all the data} (calibration and usage) of 70\% of the recordings (randomly selected) will be used as a \textit{training set}, 10\% of the recordings (randomly selected) will be used as a \textit{validation set} and the last 20\% of the recordings will be used as a \textit{test set}.
For the evaluation (testing), only the \textit{usage data} of the recordings will be used, such that the results can be cleanly compared with IAmMuse.
This evaluation will be referred to as "\textit{MARS large self trained}".\\
% MARS other user trained
\textbf{Fourth}, the MARS model is trained (in accordance with the specifications on their GitHub page \cite{mars_github}) on the data of \textbf{all but one user}, with an 80/20 train validation split.
The model will then be evaluated on the data of the user who did \textbf{not appear} in the \textit{training data}, this will be done separately for \textbf{each user}.
This evaluation will be referred to as "\textit{MARS user by user}".\\
% MARS std free
\textbf{Fifth}, the MARS model is trained (in accordance with the specifications on their GitHub page \cite{mars_github}) on the data of \textbf{all standard movement set} recordings, with an 80/20 train validation split.
The model will then be evaluated using the \textit{usage data} portion of the \textbf{free play} recordings.
This evaluation will be referred to as "\textit{MARS standard to free}".\\
% MARS all_users_representative
\textbf{Sixth}, the MARS model is trained (in accordance with the specifications on their GitHub page \cite{mars_github}) on the data of all the \textbf{first two standard recordings} of each user.
The model will then be evaluated using the \textit{usage data} portion of the \textbf{third standard recording} of each user.
This evaluation will be referred to as "\textit{MARS partial standard trained}".\\

\subsection*{in short}
\subsubsection*{Formal data specification}
Each recording contains a number of frames (\textasciitilde100) at the
start which my system uses for initialization, these frames are called
the \textbf{calibration data}. The rest of the data (which is all
\textbf{later} then the \emph{calibration data}) is called the
\textbf{usage data}.

Each \textbf{user} has \emph{multiple recordings}, they always have 3
\textbf{standard recordings} (where they performed a standardized set of
moves). Each \textbf{user} has \emph{between one and four} \textbf{free
recordings} (where they where free to move however they wanted).

My system has been \textbf{calibrated} on the \emph{calibration data} of
the specific recording in question, and has \textbf{evaluated} the
\emph{usage data} of the specific recording in question.

\subsubsection*{Formal MARS train/test/validate split}

The \textbf{MARS} system has to be trained on data in order to work, I
want to evaluate MARS with \textbf{four different trainings}. 
\begin{enumerate}
\item ``MARS pre-trained'': We use the model provided by An and Orgas in their paper on MARS to evaluate the \emph{usage data} from \textbf{all recordings}.
\item ``MARS calibration-trained'': We will train MARS ourselved with the \emph{calibration data} of \textbf{all recordings} (using an 80/20 train/test split).
This trained model will then be evaluated on the \emph{usage data} of \textbf{all recordings} .
\item ``MARS standard to free'': We will train MARS on \emph{all data} from the \textbf{standard recordings} (using an 80/20 train/test split). 
This trained model will then be evaluated on the \emph{usage data} of \textbf{all free recordings}. 
\item ``MARS partial standard trained'': We will train MARS on \emph{the first and second} \textbf{standard recording} of \textbf{each user} (using an 80/20 train/test split). 
This trained model will then be evaluated on the \emph{usage data} of \textbf{the third standard recording} of \textbf{each user}.
\end{enumerate}

