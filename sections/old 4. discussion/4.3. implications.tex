\section{Implications/final verdict}
\label{section: discussion - implications}

This thesis shows that simple algorithmic MMWave data interpretation systems, such as IAmMuse, can compete with similar simple DL interpretation systems.
Algorithmic systems are also less prone to overfitting, because the system cannot "learn" to focus on specific info which is only tangentially related to the outcome (such as the general position of the pointcloud, with the position of both arms.
Issues can also be better analyzed and fixed, because each step of the system can be inspected and tuned by a human.

\textbf{verdict}: There is real, unexplored potential in interpretation systems built on thought-out, logical algorithms.


% If my system performed better then mars then I will make a case that stochastic methods should be researched more, due to the potential quality.

% If the systems perform very differently on different people, but my system still performs well on some people I will talk about the learning curve of this system (likening it to the learning curve on an instrument, as one of the test subjects also mentioned), and still mention that further research would be beneficial.

% If my system performs poorly, I will discuss why this is most likely to be the case, I will also try to argue that it isn't likely that a different approach would change much in that respect, and I will conclude that stochastic methods don't work well for MMWave interpretation.
